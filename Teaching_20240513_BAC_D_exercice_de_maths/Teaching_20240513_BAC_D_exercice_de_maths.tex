\documentclass[12pt, openany]{book}
\renewcommand{\contentsname}{Table des matières}
\usepackage{polyglossia} %to set a language
\setmainlanguage{french}
\setotherlanguage{english}
\usepackage{hyperref}
\hypersetup{
	colorlinks=true,
	linkcolor=black,
	urlcolor=blue,
	citecolor=black,    
	pdftitle={},
	pdfauthor={Emmanuel ASSOHOU},
	pdfsubject={},
	pdfkeywords={},
}
\usepackage{xcolor} 
\usepackage{fontspec} %to specify a font
\usepackage{unicode-math} %to use all the math operators and symbols
\setmainfont{Times New Roman}
\setmathfont{Latin Modern Math}
\usepackage{enumitem} %to provide enhanced control over lists, allowing to customize the appearance and formatting of various types of lists
\usepackage{graphicx} %to use pictures in the document
\usepackage{setspace}
\setstretch{1.5}
\usepackage{tikzpagenodes} %provides convenient ways to work with page related coordinates and dimensions in TikZ
\usepackage[margin=2cm]{geometry}
\usepackage{ragged2e} %to provide enhanced justification commands for text alignment
\usepackage{amsmath} %ptovides advanced mathematical typesetting features and environments 
\usepackage{float}
\usepackage{emoji}
\usepackage{mfirstuc} %for capitalizing
\usepackage{stmaryrd}

\usepackage{tikz}
\usepackage{pgfplots}
\pgfplotsset{compat=1.15}
\pgfplotsset{soldot/.style={only marks,mark=*, line width=0.5pt, mark size=2pt}}
\pgfplotsset{holdot/.style={fill=white,only marks,mark=*, line width=1pt, mark size=2pt}}
\renewcommand{\contentsname}{Table des matières}
\usepackage{polyglossia} %to set a language
\setmainlanguage{french}
\setotherlanguage{english}
\usepackage{hyperref}
\hypersetup{
	colorlinks=true,
	linkcolor=black,
	urlcolor=blue,
	citecolor=black,    
	pdftitle={},
	pdfauthor={Emmanuel ASSOHOU},
	pdfsubject={},
	pdfkeywords={},
}
\usepackage{xcolor} 
\usepackage{amsmath}
\usepackage{longtable}
\usepackage{listings} % Include the listings package


\begin{document}
	$$\left(\ln(f)\right)'=\dfrac{f'}{f} \ , \ \left(\sqrt{f}\right)'=\dfrac{f'}{2\sqrt{f}} \ , \ \ln(a\times b)= \ln(a)+\ln(b) \,\ \ln\left(\dfrac{a}{b}\right)=\ln(a)-\ln(b)$$
	\textbf{Formule de l'intégration par parties :} $\displaystyle{\int_{a}^{b}u'(x)v(x)\ dx=\left[u(x)v(x)\right]_a^b-\int_{a}^{b}v'(x)u(x)}\ dx$\\
	
	Pour $f(x)=\ln\left(x+\sqrt{x^2+2}\right)$; on a :
	\begin{align}
		f'(x)&=\dfrac{1+\dfrac{2x}{2\sqrt{x^2+2}}}{x+\sqrt{x^2+2}}\\
		&= \dfrac{\dfrac{2\sqrt{x^2+2}+2x}{2\sqrt{x^2+2}}}{x+\sqrt{x^2+2}}\\
		&=\dfrac{2\left(\sqrt{x^2+2}+x\right)}{2\sqrt{x^2+2}\left(x+\sqrt{x^2+2}\right)}\\
		f'(x)&=\dfrac{1}{\sqrt{x^2+2}}
	\end{align}
	
	Par conséquent, pour $I=\displaystyle{\int_{0}^{1}\dfrac{1}{\sqrt{x^2+2}} \ dx}$ on en déduit que :
	\begin{align}
		I&=\displaystyle{\int_{0}^{1}\dfrac{1}{\sqrt{x^2+2}}\ dx=\int_{0}^{1}f'(x)\ dx}\\
		&= \left[\ln\left(x+\sqrt{x^2+2}\right)\right]_0^1\\
		&=\ln(1+\sqrt{3}) - \ln(\sqrt{2})\\
		I&=\ln\left(\dfrac{1+\sqrt{3}}{\sqrt{2}}\right)=\ln\left(\dfrac{\sqrt{2}+\sqrt{6}}{2}\right)
	\end{align}
	
	La moyenne de la fonction $f$ définie sur $[0;1]$ par $x\mapsto\ln\left(x+\sqrt{x^2+2}\right)$ serait :
	$$\dfrac{f(0)+f(1)}{2}=\dfrac{\ln\sqrt{2}+\ln(1+\sqrt{3})}{2}=\dfrac{\ln(\sqrt{2}+\sqrt{6})}{2}$$
	
	De plus :
	\begin{align}
		J+2I&=\displaystyle{\int_{0}^{1}\dfrac{x^2}{\sqrt{x^2+2}}\ dx+2\int_{0}^{1}\dfrac{1}{\sqrt{x^2+2\ dx}\\
		&=\displaystyle{\int_{0}^{1}\dfrac{x^2}{\sqrt{x^2+2}}\ dx+\dfrac{2}{\sqrt{x^2+2}}\ dx}\\
		&=\displaystyle{\int_{0}^{1}\dfrac{x^2+2}{\sqrt{x^2+2}}\ dx}\\
		&=\displaystyle{\int_{0}^{1}\dfrac{\sqrt{x^2+2}^2}{\sqrt{x^2+2}}\ dx}\\
		&=\displaystyle{\int_{0}^{1}\sqrt{x^2+2}\ dx}\\
		J+2I&=K
	\end{align}

	En intégrant par parties $K=\displaystyle{\int_{0}^{1}\sqrt{x^2+2}\ dx}$ on a :
	\begin{align}
		K&=\displaystyle{\left[x\sqrt{x^2+2}\right]_0^1 - \int_{0}^{1}\dfrac{x^2}{\sqrt{x^2+2}}\ dx}\\
		K&=\sqrt{3}-J
	\end{align}
	
	En posant $K=K$, on obtient des questions précédentes : 
	\begin{align}
		J+2I&=\sqrt{3}-J\\
		2J+2I&=\sqrt{3}\\
		J+I&=\dfrac{\sqrt{3}}{2}
	\end{align}
	Comme $I=\ln\left(\dfrac{\sqrt{2}+\sqrt{6}}{2}\right)$, on peut conclure que :
	$$J=\dfrac{\sqrt{3}}{2}-\ln\left(\dfrac{\sqrt{2}+\sqrt{6}}{2}\right)$$
	
	Donc $K=\sqrt{3}-\left(\dfrac{\sqrt{3}}{2}-\ln\left(\dfrac{\sqrt{2}+\sqrt{6}}{2}\right)\right)=\dfrac{\sqrt{3}}{2}+\ln\left(\dfrac{\sqrt{2}+\sqrt{6}}{2}\right)$
	
\end{document}