\newpage \begin{center}
	\subsubsection*{Emmanuel Macron, président de la République française affirme : \textit{« Je n’aime pas le terme (de pénibilité) car il induit que le travail est une douleur »}. Pensez-vous que le travail est nécessairement une douleur ? }\addcontentsline{toc}{subsection}{20240226 - ISE 2018 (Sujet 3)}
	$\star \star \star$
\end{center}

Depuis fort longtemps, l'esprit humain a toujours associé la notion de travail à la dépense physique, intellectuelle qui entraine un sentiment de pénibilité. Cependant, la tendance tend à s'inverser car de nos jours, certains considère le travail comme source d'épanouissement. Dès lors, il devient important de se questionner sur les effets du travail sur l'homme.

La douleur pourrait se définir comme la manifestation particulièrement intense de la pénibilité. Le travail se définirait quant à lui comme l'ensemble des activités physiques, intellectuelles ou mixtes, considéré comme un facteur essentiel dans la production.

En raison des défintions précédentes, l'on peut se poser les différentes questions suivantes : Est-ce que le travail et la douleur seraient liés ? Quels sont les facteurs qui emmèneraient à qualifier le travail de douleur ? Le travail, ne pouvant être obligatoirement considéré comme une douleur, nous amène à nous donc à réfléchir sur les états émotionnels que procurent le travail.

Est-ce que le travail peut-être toujours associé à la douleur ?

Bien que le travail soit considéré comme pénible pour certains (I), celui-ci serait un moyen d'épanouissement pour d'autres.
$$\star \star \star$$

\begin{center}
	\textbf{\underline{PLAN D\'ETAILL\'E}}
\end{center}
\noindent I) Le travail induirait de la douleur.

\indent 1) Le travail induirait une douleur du fait de l'effort physique qu'il demande.

\indent 2) Le travai induirait une douleur du fait de ses répercussions sur le cerveau.

\indent \indent \underline{EX :} Le stress, l'anxiété, les problèmes de santé mentale.\\

\noindent II) Le travail source de l'épanouissement.

\indent 1) Le travail comme moyen pour l'individu de démonstration de son utilité pour la société.

\indent 2) Le travail comme moyen d'enrichissement intellectuel.

