
\newpage %\begin{center}
	\subsubsection*{\textit{Après avoir décrit de manière théorique les avantages et les inconvénients des politiques protectionnistes mises en place dans de nombreux pays ou Unions régionales, vous analyserez les pratiques protectionnistes de certains pays ou Unions de l'Afrique.}}\addcontentsline{toc}{subsection}{20240312 - ISE 2023 Sujet (Eco-1)}
	%20240228 - ISE 2018 Sujet (Eco-1)
	%20240226 - ISE 2018 (Sujet 3)
%\end{center}

\textbf{Problématisation :} 

Dans ce sujet, le terme \guillemetleft Unions\guillemetright\ attire notre attention car il présuppose, d'une part, l'existence de plusieurs types de blocs formés de pays. D'autre part, ces blocs se réunissent autour d'un objectif commun consistant le plus souvent en un renforcement de relation. En outre, ces relations sont souvent d'ordre économique, commerciale, monétaire, politique, sécuritaire etc. Mais, la notion de \guillemetleft pratiques protectionnistes\guillemetright\ faisant allusion à un ensemble de politiques commerciales restrictives, met mieux en avant la notion de regroupement étatique commercial. Cependant, le mode de fonctionnement commercial de ces unions pourrait différer d'une région à une autre. Dès lors, l'on peut se poser la question suivante :

Quelle est la particularité des politiques protectionnistes des unions Africaines ?

 