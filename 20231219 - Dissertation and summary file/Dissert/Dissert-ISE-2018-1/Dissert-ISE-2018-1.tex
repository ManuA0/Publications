\begin{center}
	\subsection*{Pour l’écrivain français Paul Valéry \textit{« Tout état social exige des fictions »}.  Selon vous, une société a-t-elle besoin de fictions, de rêves, d’utopies ? }\addcontentsline{toc}{subsection}{20240123- ISE 2018 (sujet 1)}
\end{center}

Tout état social exige des fictions écrit Paul Valéry. Cette idée montre la nécessité pour les structures sociales de reposer sur des concepts imaginaires. Dès lors, l'on pourrait se demander si fictions, rêves et utopies ne seraient-ils pas un must dans une société.

En sociologie, un état social se définirait comme l'organisation sociale d'une communauté, comprenant ses structures, institutions et relations interpersonnelles. Cependant, selon le droit il désignerait le statut social d'une personne, indiquant sa position dans la hiérarchie sociale. Les fictions, les rêves et utopies quant à eux seraient des manifestations de l'imagination humaine. 

Au regard des nombreuses sociétés d'antan au nombre desquelles figure la société grecque dont le fonctionnement sociétal était basé sur la vision de grands philosophes de l'époque, l'on pourrait se demander si société et imagination sont liées. En effet, les écrits des grands écrivains de ce temps émanant de leur imagination d'une cité parfaite dictaient les classes sociales qui devaient exister et leurs rôles afin d'assurer l'harmonie dans la cité. Par conséquent, pourrait-on dire que la société serait basée sur l'imagination ?

L'imagination humaine serait-elle l'unique besoin d'une société ?

Bien que la société reposerait sur l'imagination dans certains cas (I), il existerait, cependant, d'autres facteurs nécessaires pour la société au-delà de l'imagination (II).

$$\star \star \star$$

La société reposerait de prime abord sur l'imagination. 

En effet, l'imagination serait utile pour l'avancement de la société car elle serait l'essence du progrès scientifique et social. D'une part, car dans le processus de création, tout commence par la pensée et la vision de l'homme relative à des solutions pour un problème donné. Et c'est donc la résolution de ce problème qui permet à la société d'aller de l'avant, d'où le rôle essentiel de l'imagination dans la société. C'est en ce sens que Albert Einstein la qualifiera d'importante dû fait qu'elle stimule le progrès et suscite l'évolution.

D'autre part, elle serait la cause du progrès social. En effet, issu d'un désir humain d'un changement dans la composition de la société, le progrès social serait entrainé par la création d'un imaginaire de vie qui lui même serait influencé par de nombreux facteurs tels que la situation comtemporaine de vie, le temps etc. C'est ainsi qu'auparavant il existait la classe des hommes libres et des esclaves puis uniquement la classe des hommes libres, de nos jours, depuis l'abolition de l'esclavage et de la colonisation due aux combats d'hommes pour la concrétisation de leur imaginaire où tous seraient égaux.

Face à de tels impacts de l'imagination sur la société, il est par conséquent important de se demander si l'imagination serait le seul moteur de la société.

$$\star \star \star$$

Outre l'imagination, la société aurait aussi besoin d'éducation et de critiques pour la faire avancer.

De prime abord, l'éducation apparait comme un besoin primordial pour la société car ce sont les guerres à répétiton dans certaines régions du tiers monde, la pauvreté et bien d'autres facteurs qui ont permis de mettre en évidence le rôle de l'éducation dans le développement de la société. Parceque contrairement aux pays développés, ces pays du tiers monde ont des systèmes éducatifs défaillants et de forts taux d'analphabétisation. C'est en ce sens que parmi les trois composantes de l'indice de développement humain (IDH) d'un pays on retrouve l'éducation.

Par ailleurs, les critiques joueraient un grand rôle dans la société car elles seraient d'une part, le moyen de mettre en évidence les failles de la société et d'autre part, le lieu des recommandations pour l'avancement de la société. En effet, ce sont les observations de l'opposition, dans le sytème politique démocratique, qui permettent aux politiciens en exercice de remarquer leurs erreurs, de les corriger et de prendre des décisions plus intelligentes pour l'avenir.

Dès lors, éducation et critiques sont tout aussi des besoins importants pour la société.

$$\star \star \star$$

En définitive, l'imagination humaine (fictions, rêves, utopies) sont, au même titre que l'éducation et les critiques, des besoins primordiaux pour la société. Car ils favoriseraient l'avancement de la société sur le plan scientifique, social, politique etc.

\newpage \begin{center}
	\textbf{PLAN D\'ETAILL\'E}
\end{center}

\noindent \textbf{La société reposerait sur l'imagination dans certains cas (I).}
\begin{itemize}
	\item Car l'imagination est l'essence du progrès scientique dans une société.
	\item Car l'imagination est la source du  progrès social.
\end{itemize}

\noindent \textbf{au-delà de l'imagination, il existerait d'autres facteurs nécessaires pour la société (II).}

\begin{itemize}
	\item Car outre l'imagination, l'éducation serait nécessaire pour une société
	\item Les citiques sont toutes aussi utiles pour une société
\end{itemize}