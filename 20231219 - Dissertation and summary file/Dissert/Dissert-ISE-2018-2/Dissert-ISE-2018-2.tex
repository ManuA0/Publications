\newpage \begin{center}
	\subsection*{Léopold Sédar Senghor, poète et homme d’état sénégalais a dit: \guillemetleft \textit{La francophonie, c’est cet humanisme intégral qui se tisse autour de la terre} \guillemetright. Qu’en pensez-vous ? }\addcontentsline{toc}{subsection}{20240212 - ISE 2018 (Sujet 2)}
\end{center}

Pour certains auteurs comme Léopold Séda Senghor, le but de la francohponie irait au-delà d'un rassemblement d'états repartis sur les cinq continents ayant le français comme langue commune. C'est en ce sens que l'homme d'état sénégalais affirmait que la francophonie est un humanisme qui se tisse autour de la terre. Dès lors, il serait important de se questionner sur les raisons qui l'ont motivé à dire cela.

La francophonie se définit par l'organisation internationale de la francophonie, d'une part comme l'ensemble des femmes et hommes (estimé au total à trois cents vingt millions de locuteurs selon le rapport en date de l'Observatoire de la langue française, publié en 2018) partageant une langue commune, le français. D'autre part, elle désignerait un dispositif institutionnel voué à promouvoir le français et à mettre en œuvre une coopération politique, éducative, économique et culturelle au sein de quatre vingts huit états. L'humanisme est quant à lui défini du point de vu l'histoire et de la philosophie. En effet, il désignerait selon du point de vu philosophique, une théorie ou une doctrine mettant la personne humaine et son épanouissement au-dessus de toutes les autres valeurs. Tandis que sur le plan historique, il fait référence à un mouvement de la Renaissance caratérisé par un effort pour relever la dignité de l'esprit humain et le mettre en valeur.

Dès lors, dans quelle mesure la francophonie, en tant que concept alliant diversité linguistique et culturelle, peut-elle véritablement incarner un humanisme intégral selon la vision de Léopold Sédar Senghor ? Quelles sont les implications de cette conception pour les relations interculturelles, la coopération internationale et la promotion des valeurs humanistes dans un monde en mutation ?

Dans quelle mesure la francophonie représenterait un humanisme intégral et international ?

Bien que la francophonie soit considérée comme un vecteur de diversité culturelle et linguistique (I), elle pourrait aussi être vu comme un espace de dialogue et de coopération internationale (II).
$$\star \star \star$$

La diversité culturelle, premier pilier de la francophonie, se manifeste à travers une multiplicité de traditions, coutumes et arts propres à chaque pays francophone. Cette diversité, d'une part, enrichit la palette culturelle mondiale et, d'autre part, constitue un levier puissant pour la valorisation des patrimoines culturels locaux. En mettant en lumière ces richesses, la francophonie promeut l'épanouissement des identités nationales tout en favorisant un sentiment d'appartenance communautaire plus large.

Dans le même temps, la francophonie favorise la promotion de la diversité culturelle par le biais d'échanges culturels et artistiques entre ses membres. Ces interactions dynamiques permettent non seulement une meilleure compréhension mutuelle, mais également une célébration des différences qui caractérisent chaque société francophone. Parallèlement, la reconnaissance et la préservation des langues minoritaires au sein de la francophonie garantissent la pérennité de la pluralité linguistique et culturelle au sein de cette communauté.
$$\star \star \star$$

La francophonie, bien que porteuse d'idéaux humanistes, rencontre des défis et des limites dans sa réalisation de l'humanisme intégral.

A. Les tensions et les divisions au sein de la francophonie représentent l'un des principaux défis à surmonter. Malgré le partage de la langue française, les pays membres peuvent être confrontés à des divergences politiques, économiques ou culturelles, entraînant des tensions au sein de l'organisation. Ces divisions peuvent compromettre la capacité de la francophonie à agir de manière unie et cohérente sur la scène internationale, limitant ainsi son efficacité en tant qu'outil de promotion des valeurs humanistes.

B. Parallèlement, la mondialisation pose des défis à la préservation de l'identité culturelle au sein de la francophonie. Alors que les échanges culturels et économiques s'intensifient à l'échelle mondiale, les cultures nationales risquent d'être submergées par une culture mondiale uniforme. La francophonie doit donc trouver des moyens efficaces de protéger et de promouvoir la diversité culturelle de ses membres dans un monde de plus en plus globalisé, afin de préserver l'expression de leurs identités uniques et la richesse de leurs patrimoines culturels.

\begin{center}
	\textbf{Plan détaillé}
\end{center}
I) Présentation de la 
