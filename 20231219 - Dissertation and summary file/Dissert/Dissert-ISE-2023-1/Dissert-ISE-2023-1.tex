\begin{center}
	\subsection*{\textit{Conquérir l'espace, est-ce bien raisonnable ?}}\addcontentsline{toc}{subsection}{20230920 - ISE 2023 (sujet 1)}
\end{center}
\hspace*{0.5cm}La fascination pour l’exploration spatiale a de tout temps animé l’humanité. Depuis les premiers pas sur la Lune jusqu'aux projets ambitieux d'envoyer des humains sur Mars, la conquête de l'espace est devenue une réalité qui suscite à la fois l'admiration et l'interrogation. Au-delà de la quête de nouvelles frontières et de la recherche scientifique, se pose la question fonda-mentale de la rationalité de ces entreprises. \\
\hspace*{0.5cm}L’expression verbale « conquérir l'espace » pourrait être définie comme l'ensemble des activités humaines visant à explorer, coloniser et exploiter des régions ou planètes situées au-delà de la Terre, notamment la Lune, Mars etc.\\
\hspace*{0.5cm}Au regard des sommes astronomiques dédiées par certains États et entreprises à la conquête de l’espace et des nombreuses dénonciations des conséquences liées à cette activité sur notre planète lorsque la pauvreté extrême sévit dans certaines régions ; il est par conséquent important de se demander s’il est réellement raisonnable de consacrer d'énormes ressources financières et technologiques à l'exploration de l'espace.\\
\hspace*{0.5cm}Dans quelle mesure la conquête de l’espace serait-elle rationnelle ?\\
\hspace*{0.5cm}Bien que la conquête spatiale soit avantageuse pour le développement technologique et la survie de l’espèce humaine (I), elle n’en resterait cependant dénuée de risques pour notre planète terre (II).
$$\star \star \star$$
\hspace*{0.5cm}La conquête spatiale parait de prime abord bénéfique. En effet, elle serait avantageuse pour l’avancée technologique, l’innovation et l’industrie. D’abord, elle favoriserait l’avancée technologique car n’étant pas essentiellement captée par les voyages spatiaux, elle consisterait aussi au lancement de satellites tournant autour de notre planète qui ont permis de depuis plu-sieurs années à améliorer nos moyens de communication, de localisation et aussi de prévention de catastrophes naturelles. Ensuite, elle apporterait une panoplie d’innovations qui permettrait aux États de pouvoir faire des économies d’échelle et à bon nombre de secteurs tels que la médecine, l’automobile…qui tirent profit des découvertes des nombreuses recherches spatiales. C’est le cas de la pompe d’assistance ventriculaire utilisée dans les cœurs artificiels qui est un produit découlant des pompes de la navette spatiale américaine conçues en 2000. Enfin, elle serait un stimulus pour le secteur industriel d’une part parce que ces nombreuses découvertes gagneraient à être mises en forme afin d’être commercialisées au grand public et d’autre part, du fait de ses nombreuses découvertes, la conquête spatiale permettraient permettrait à certains riches capitalistes de générer emplois en créant de nouvelles industries concurrentes ; ce, dans le but de détruire les monopoles industriels préjudiciables pour l’harmonie économique.\\
\hspace*{0.5cm}Même si la conquête spatiale favorise l’avancée technologique et l’innovation nécessaires pour le bon fonctionnement des industries et de nos économies, peut-elle être considérée sans danger.
$$\star \star \star$$
\hspace*{0.5cm}La conquête de l’espace ne pourrait toutefois être sans conséquences. Les couts financiers énormes liés à la conquête spatiale serait la cause de la diminution des fonds alloués à d’autres domaines de l’économie et nécessaires pour relever les défis terrestres. En raison de la limitation des ressources, les États enclin à la compétition spatiale délaisserait certains pans de l’économie comme ce fut le cas durant la période de guerre froide en URSS. De plus, elle aurait des couts énormes sur l’écologie. Puisqu’elle nécessite d’éjecter des fusées fabriquées dans des industries à empreinte carbone importante à l’extérieur de l’espace terrestre ; participant à la destruction de la couche d’ozone qui elle-même nous protège des rayons ultraviolets du soleil. La conquête spatiale serait donc nocive pour l’équilibre écologique et participerait à la hausse de température dans certaines régions du globe. Outre, les couts subits par notre planète terre et ses habitants, l’exploration spatiale constituerait un cout humain vu que le voyage interstellaire constituerait un risque lié à l’incertitude de ce que les astronautes pourraient trou-ver sur les planètes qu’ils visitent. De plus l’exposition directe et longue à l’impesanteur néfaste pour la circulation sanguine, et aux lumières du soleil, étoiles favorisant la mutation génétique, l’affaiblissement de la vision et la perte de masse osseuse seraient les quelques dangers auxquels font face les cosmonautes en orbite que l’on peut retenir de l’article de \textit{Kheira BETTAYEB} publié le 06 mars 2022 dénommé « Conquête spatiale : quelles sont les limites humaines ? ».\\
\hspace*{0.5cm}Dès lors, la conquête spatiale n’est pas sans risques.
$$\star \star \star$$
\hspace*{0.5cm}En somme, s’interroger sur la rationalité de la conquête spatiale nous a permis de nous rendre compte que celle-ci est bénéfique pour notre planète car elle favorise l’avancée technologique, l’innovation et stimule l’industrie nécessaire pour la bonne marche de l’économie. Aussi, nous a-t-il permis de nous rendre compte les couts financiers, écologiques et humains qu’elle nécessite pour son accomplissement font d’elle une activité risquée pour l’homme et pour la planète à laquelle il devrait s’y adonner avec précaution.\newpage