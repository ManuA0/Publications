\newpage %\begin{center}
	\subsubsection*{Dans un document intitulé \guillemetleft \textit{Les infrastructures à l’horizon 2030}\guillemetright,  l’OCDE note que : \newline \guillemetleft Les réseaux d’infrastructure jouent un rôle vital dans le développement économique et social. De plus en plus interdépendants, ils constituent un moyen d’assurer la fourniture et la prestation de biens et de services qui concourent à la prospérité et à la croissance économique et contribuent à la qualité de vie. La demande d’infrastructure est appelée à sensiblement augmenter dans les décennies à venir, sous l’impulsion de facteurs majeurs de changement comme la croissance économique mondiale, le progrès technologique, le changement climatique, l’urbanisation et l’intensification de la congestion. Toutefois, les défis à relever sont multiples (…)\guillemetright \newline \newline Après avoir rappelé les théories relatives à la croissance, vous préciserez le rôle des infrastructures et les défis auxquels sont confrontées les nations à différents niveaux de croissance pour en assurer la pérennité et le développement. }\addcontentsline{toc}{subsection}{20240228 - ISE 2018 Sujet (Eco-1)}
%\end{center}

 \noindent \textbf{Problématisation :} 