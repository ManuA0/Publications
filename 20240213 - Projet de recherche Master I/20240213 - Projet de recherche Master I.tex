\documentclass[12pt,a4paper, openany]{book}
\renewcommand{\contentsname}{Table des matières}
\usepackage{polyglossia} %to set a language
\setmainlanguage{french}
\setotherlanguage{english}
\usepackage{hyperref}
\hypersetup{
	colorlinks=true,
	linkcolor=black,
	urlcolor=blue,
	citecolor=black,    
	pdftitle={},
	pdfauthor={Emmanuel ASSOHOU},
	pdfsubject={},
	pdfkeywords={},
}
\usepackage{xcolor} 
\usepackage{fontspec} %to specify a font
\usepackage{unicode-math} %to use all the math operators and symbols
\setmainfont{Times New Roman}
\setmathfont{Latin Modern Math}
\usepackage{enumitem} %to provide enhanced control over lists, allowing to customize the appearance and formatting of various types of lists
\usepackage{graphicx} %to use pictures in the document
\usepackage{setspace}
\setstretch{1.5}
\usepackage{tikzpagenodes} %provides convenient ways to work with page related coordinates and dimensions in TikZ
\usepackage[margin=2cm]{geometry}
\usepackage{ragged2e} %to provide enhanced justification commands for text alignment
\usepackage{amsmath} %ptovides advanced mathematical typesetting features and environments 
\usepackage{float}
\usepackage{emoji}
\usepackage{mfirstuc} %for capitalizing
\usepackage{stmaryrd}

\usepackage{tikz}
\usepackage{pgfplots}
\pgfplotsset{compat=1.15}
\pgfplotsset{soldot/.style={only marks,mark=*, line width=0.5pt, mark size=2pt}}
\pgfplotsset{holdot/.style={fill=white,only marks,mark=*, line width=1pt, mark size=2pt}}
\renewcommand{\contentsname}{Table des matières}
\usepackage{polyglossia} %to set a language
\setmainlanguage{french}
\setotherlanguage{english}
\usepackage{hyperref}
\hypersetup{
	colorlinks=true,
	linkcolor=black,
	urlcolor=blue,
	citecolor=black,    
	pdftitle={},
	pdfauthor={Emmanuel ASSOHOU},
	pdfsubject={},
	pdfkeywords={},
}
\usepackage{xcolor} 
\usepackage{amsmath}
\usepackage{longtable}
\usepackage{listings} % Include the listings package

\fancyfoot[C]{\thepage}

\begin{document}

\frontmatter
%%%%%%%%%%%%%%%%%%%%%%%%%%%%%%%%%%%%%%%%%%%%%%%%%%%%%%%%%%%%%%%%%%%%%%%%%%%%%%%%%%%%%%%%%%%%%%%%%%%%%%%%%%%%%%%%%%%%%%%%%%%%%%%%%%%%%%%%%%%%%%%%%%%%%%%%%%%%%%%%%%%%%%%%%%%%%%%%%%%%%%%
	% Make a table with two columns
	\begin{longtable}{p{0.5\textwidth}p{0.4\textwidth}}
		\centering \includegraphics[width=5.77cm, height=2.5 cm]{"C:/Users/Emmanuel/Documents/latex-files/Photos/Coverpage/téléchargement (1).jpg"} \newline  \small{\textbf{UFR DES SCIENCES \'ECONOMIQUES ET D\'EVELOPPEMENT (UFR-SED)}}  & \centering \hspace{2.5cm}\includegraphics[width=3.33 cm, height=2.91 cm]{"C:/Users/Emmanuel/Documents/latex-files/Photos/Coverpage/Image2.png"} \newline \hspace*{2cm}République de Côte d’Ivoire \newline \hspace*{1.85cm}\textit{Union – Discipline – Travail}\\
		\endhead
		\hline
		\multicolumn{2}{p{\textwidth}}{
			\begin{center}
					\textbf{TECHNIQUES D'\'ELABORATION D'UN PROJET DE RECHERCHE}\\
					\textbf{\underline{THÈME :}}\\ \textbf{INFLUENCE DE L'IDENTITÉ DES JEUNES SUR LEUR EMPLOYABILITÉ \newline EN CÔTE D'IVOIRE.}
			\end{center} 
			\begin{center}
				\noindent \textcolor{blue}{Année académique 2023-2024}
			\end{center}  
			\begin{center}
				\noindent MASTER 1 POLITIQUE ÉCONOMIQUE ET MODÉLISATION
			\end{center}
		}  \\
		\hline
	\end{longtable}
%%%%%%%%%%%%%%%%%%%%%%%%%%%%%%%%%%%%%%%%%%%%%%%%%%%%%%%%%%%%%%%%%%%%%%%%%%%%%%%%%%%%%%%%%%%%%%%%%%%%%%%%%%%%%%%%%%%%%%%%%%%%%%%%%%%%%%%%%%%%%%%%%%%%%%%%%%%%%%%%%%%%%%%%%%%%%%%%%%%%%%%

	\textit{PR\'ESENT\'E PAR}: 
	
	% Participants list
	\begin{longtable}{|p{0.02\textwidth}|p{0.19\textwidth}|p{0.15\textwidth}|p{0.13\textwidth}|p{0.41\textwidth}|}
		\hline 
		\centering \textbf{N°}&
		\centering \textbf{Identifiant permanent}&
		\centering \textbf{N° Carte d’étudiant}&
		\centering \textbf{Nom} &
		\multicolumn{1}{c|}{\textbf{Prénoms}}\\ \hline \hline
		\endhead % This indicates the header should repeat on each page for long tables
		1&  AHOA0205040001& CI0321003447&  \multicolumn{1}{c|}{AHOUSSI}& \multicolumn{1}{c|}{AHOUSSI JEAN-CHARLES ESDRAS}\\ \hline 
		2&  ASST0202030001& CI0321003010&  \multicolumn{1}{c|}{ASSOHOU}& \multicolumn{1}{c|}{TANO FRANCK EMMANUEL}\\ \hline
		3&  GNAA1211020001& CI0321002906&  \multicolumn{1}{c|}{GNAHOR\'E}& \multicolumn{1}{c|}{AMOS}\\ \hline
		4&  GNAW2808030001& CI0321002317&  \multicolumn{1}{c|}{GNAHOUA}& \multicolumn{1}{c|}{WILFRIED}\\ \hline
		%5&  TOTK2303030001& CI0321000242&  \multicolumn{1}{c|}{TOTO}& \multicolumn{1}{c|}{KOUAM\'E BAH \'ELICHAMA}\\ \hline
	\end{longtable}
%%%%%%%%%%%%%%%%%%%%%%%%%%%%%%%%%%%%%%%%%%%%%%%%%%%%%%%%%%%%%%%%%%%%%%%%%%%%%%%%%%%%%%%%%%%%%%%%%%%%%%%%%%%%%%%%%%%%%%%%%%%%%%%%%%%%%%%%%%%%%%%%%%%%%%%%%%%%%%%%%%%%%%%%%%%%%%%%%%%%%%%
	% Teacher's name
	\begin{longtable}{p{\textwidth}}
		\\
		\\
		\\
		\\
		\\
		\\
		\\
		\centering Pr. Wadjamsse B. DJEZOU
	\end{longtable}
%%%%%%%%%%%%%%%%%%%%%%%%%%%%%%%%%%%%%%%%%%%%%%%%%%%%%%%%%%%%%%%%%%%%%%%%%%%%%%%%%%%%%%%%%%%%%%%%%%%%%%%%%%%%%%%%%%%%%%%%%%%%%%%%%%%%%%%%%%%%%%%%%%%%%%%%%%%%%%%%%%%%%%%%%%%%%%%%%%%%%%%

%	\renewcommand\cftchapteraftersnumb{\Roman} % This command redefines the appearance of chapter entries in the table of contents. Specifically, it sets the font to \normalfont (the normal font), removing any special formatting that may have been applied to the chapter numbers.
	
%	\renewcommand\cftbeforechapterskip{5pt plus 1pt} %This command redefines the space before each chapter entry in the table of contents. It sets the space to 5 points, with some stretchability (plus 1 point). This allows a small amount of flexibility in the spacing to improve the overall appearance of the table of contents.
	
	%\clearpage % Remove blank page
	\shorttableofcontents{Sommaire}{1}
	
	\mainmatter
%%%%%%%%%%%%%%%%%%%%%%%%%%%%%%%%%%%%%%%%%%%%%%%%%%%%%%%%%%%%%%%%%%%%%%%%%%%%%%%%%%%%%%%%%%%%%%%%%%%%%%%%%%%%%%%%%%%%%%%%%%%%%%%%%%%%%%%%%%%%%%%%%%%%%%%%%%%%%%%%%%%%%%%%%%%%%%%%%%%%%%%
	%\newpage 
	%\nomenclature{OIT}{Organisation Internation du Travail}
\nomenclature{PNUD}{Programme des Nations-unies pour le Développement}
	%\printnomenclature
	%\addcontentsline{toc}{chapter}{Liste des abbréviations}
	
	\chapter{INTRODUCTION}  

\noindent Selon le Programme des Nations Unies pour le Développement (PNUD, 2013), en Côte d'Ivoire, le nombre d'étudiants entrant sur le marché du travail est 3,5 fois supérieur au nombre d'emplois disponibles \citep{franccoisuniversites}. Ce constat met en évidence un important déséquilibre entre l'offre et la demande d'emploi en Côte d'Ivoire. D'où la nécessité de se pencher sur la question de l'employabilité des jeunes dans ce pays et d'étudier les facteurs qui l'expliquent.\\

\noindent Les nations unies définissent le \guillemetleft jeune\guillemetright\ comme une personne agée de 15 à 24 ans. Cependant, cette défintion de la jeunesse est considérée comme restreinte car au sens plus large, un jeune serait une personne dont l'âge est compris entre 15 et 35 ans, voire 40 ans. Par ailleurs, l'Organisation internationale du travail (OIT), définit l'employabilité comme l'aptitude de chacun à trouver et conserver un emploi, à progresser au travail et à s'adapter au changement tout au long de la vie professionnelle. Aussi, est-il important de souligner que le concept d'\guillemetleft identité\guillemetright\ désignera, dans le cadre de notre travail, les caractéristiques éthniques, politiques et réligieuses d'un individu dans la société. De son côté, \guillemetleft une ethnie\guillemetright\ ou \guillemetleft un groupe ethnique\guillemetright\, selon Wikipédia, est une population humaine ayant en commun une origine ou une ascendance, une histoire, une mythologie, une culture, une langue ou dialecte et un mode de vie.\\

\noindent Situé en en Afrique de l'ouest entre le Mali et le Burkina Faso au nord, la Guinée et le Liberia à l'ouest, le Ghana à l'est et le Golfe de Guinée au sud, la Côte d'Ivoire a une superficie de 322 463 km$^2$ avec une population estimée en 2023 à 29 344 847 d’habitants dont 42\% sont musulmans, 34\% sont chrétiens, 16,7\% n’ont pas de croyance religieuse, et le reste de la population pratique la religion traditionnelle. Aussi, compte-t-elle plus de soixante ethnies réparties en quatre groupes: les Akans (environ 42\% de la population), les Mandés (environ 27\% de la population), les Voltaïques (environ 16\% de la population) et les Krous (environ 15\% de la population). Cependant, au delà de ces atouts naturels, culturels, réligieux et démographiques, le pays fait face depuis quelques années à une montée de la violence juvénile, une augmentation de la propagation des maladies sexuellement transmissibles telles que les IST et le VIH...chez les jeunes à cause d'une pauvreté accrue dans cette classe d'âge liée pour bon nombre d'auteurs au manque d'opportunités économiques tels que le chômage. Ce dernier résulte de plusieurs facteurs tels que la crise économique aiguë, la mauvaise gouvernance, la formation et une politique de l’emploi inadaptée \citep{simeon2018jeunes}, des attentes trop élevées des pourvoyeurs d'emploi en terme de niveau d'étude et d'expérience acquis par le demandeur d'emploi et l’inefficience du dispositif d’insertion professionnelle due en partie à des contraintes financières et institutionnelles, ainsi qu’à une insuffisance de ressources humaines adéquates \citep{n2015employabilite}. Suite à cela, l'on pourrait également mentionner la faute des jeunes dans cette situation parceque, comme le montre \cite{n2015analyse}, les jeunes ivoiriens sont moins friands à rechercher un emploi par rapport aux étrangers et s'orientent vers les concours. De surcroit, il montre encore que lorsqu'ils se trouvent à proximité d'une agence d’un service public de l’emploi (SPE), ceux-ci ont n'ont pas tendance à s'orienter vers elle pour la recherche d'un emploi. Dès lors, outre l'\'Etat, le manque de volonté des jeunes constitue de même une barrière à leur employabilité. Néanmoins, le développement récent de politiques de recrutement dans les entreprises, qui accordent une importance à l'origine ethnique ou à l'appartenance politique des jeunes candidats, susciterait l'interrogation fondamentale suivante :\\

Est-ce que l'identité est un facteur déterminant de l'accès à l'emploi pour les jeunes en Côte d'Ivoire ?\\

\noindent Ainsi, comme questions secondaires de recherche nous aurons :

\begin{enumerate}[label=\arabic*)]
	\item Est-ce que l'appartenance ethnique joue un rôle significatif dans l'accès à l'emploi pour les jeunes en Côte d'Ivoire ?
	\item Est-ce l'orientation politique a un impact significatif sur les opportunités professionnelles des jeunes dans ce pays ?
\end{enumerate}

\noindent En raison de ce qui précède, nous pouvons supposer que :
\begin{enumerate}[label=\arabic*)]
	\item Les jeunes ivoiriens appartenant à certains groupes ethniques ont des chances différentes d'accéder à des emplois.
	\item L'appartenance politique influence la perception des jeunes par les employeurs, ce qui affecterait leur employabilité. 
\end{enumerate}

\noindent Notre étude aura pour objectif de mettre en évidence les relations causales entre l'employabilité des jeunes ivoiriens et deux facteurs à savoir leur appartenance ethnique d'une part, et leur affiliation politique d'autre part, tout palliant au manque d'étude relatif à cette question dans la littérature.

\newpage
\chapter{Réponses aux questions}

\section{La problématique}
Influence de l'identité des jeunes sur leur employabilité en Côte d'Ivoire.

\section{La question de recherche}
Est-ce que l'identité est un facteur déterminant de l'accès à l'emploi pour les jeunes en Côte d'Ivoire ?

\section{Les objectifs}
Mettre en évidence les relations causales entre l'employabilité des jeunes ivoiriens et deux facteurs à savoir leur appartenance ethnique d'une part, et leur affiliation politique d'autre part, tout palliant au manque d'étude relatif à cette question dans la littérature.

\section{Les hypothèses}
\begin{enumerate}[label=\arabic*)]
	\item Les jeunes ivoiriens appartenant à certains groupes ethniques ont des chances différentes d'accéder à des emplois.
	\item L'appartenance politique influence la perception des jeunes par les employeurs, ce qui affecterai leur employabilité. 
\end{enumerate}

	\nocite{*}
	\bibliographystyle{apacite}
	\bibliography{bibliographie/bibliographie.bib}
	\addcontentsline{toc}{chapter}{Bibliographie}
	
	\tableofcontents
	
\end{document}
