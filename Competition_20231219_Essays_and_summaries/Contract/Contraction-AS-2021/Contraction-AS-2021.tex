\begin{center}
	\subsection*{\textit{L’entreprise altruiste.}}\addcontentsline{toc}{subsection}{20220729 - Contraction de texte (AS 2021)}
\end{center}
\hspace*{0.5cm}L’idée délaissée par les entreprises altruistes est la recherche éperdue de profit écono-mique. Contrairement aux entreprises capitalistes ; elles priorisent l’idée de la recherche de valeur sociale. Cela, grâce aux liens humains vrais qu’elles tissent avec leurs parties prenantes. Pour cela, elles doivent agir sur les composants de leur environnement en créant un cadre d’échange et de travail harmonieux pour toutes les personnes cohabitant ou non dans ces en-treprises. Toutefois, ce mode de fonctionnement d’entreprise reste difficilement accepté par tous.\\
\hspace*{0.5cm}Cependant, contre toute attente, l’entreprise pourrait profiter économiquement de ses améliorations sociales. En effet, malgré les souffrances physiques liées au travail industriel d’autrefois ; devenues aujourd’hui psychologiques ; l’entreprise a réussi à influencer le style de vie de son environnement extérieur. D’une part, en rendant accessible à la classe moins aisée ; certains produits qu’elle ne pouvait se permettre et d’autre part, en accordant une place impor-tante au marchand dans la création de valeur. Ainsi, agit-elle aveuglément pour l’émancipation de tous.\\
\hspace*{0.5cm}Néanmoins, malgré toute la valeur sociale qu’elles créent par leur bonne foi ; les entreprises altruistes ne seraient la solution à tous les problèmes sociaux. Parce que, même si elles ont pour but primaire d’apporter de la valeur sociale au détriment du résultat économique ; leur méthode d’action reste critiquée par bon nombre d’auteurs éminents. Car pour eux, l’aide de ces entreprises n’améliore aucunement la condition sociale des prolétaires et n’a donc aucun effet positif sur leur style de vie. En dépit, des critiques qu’elles subissent ; elles réussissent quand même à résoudre certains problèmes, grâce leur idéologie centrée sur l’humain.
$$\star \star \star$$
\begin{center}
	\underline{\textbf{274 mots}}
\end{center}\newpage
