\newpage %\begin{center}
	\subsubsection*{Le monde du travail est en mutation, sous l’effet de la numérisation de l’économie et du changement technologique global. Ces processus, associés à la mondialisation et les changements dans l’organisation du travail vont façonner le monde du travail et poseront des défis inédits aux politiques publiques. \newline \newline Après avoir rappelé les différentes théories permettant d’expliciter les dysfonctionnements du marché du travail, vous présenterez les principaux dilemmes des politiques de l’emploi.
	}\addcontentsline{toc}{subsection}{20230313 - ISE 2018 Sujet (Eco-2)}
	%date - ISE 20... Sujet (Eco-1)
	%date - ISE 20... (Sujet 3)
%\end{center}
$$\star \star \star$$

%\noindent \textbf{Définition des termes du sujet :}
%
%\begin{itemize}
%	\item 
%\end{itemize}

\begin{center}
	\textbf{\underline{Introduction}} 
\end{center}

L'utilisation des groupes verbaux suivants \guillemetleft vont façonner le monde travail\guillemetright\ et \guillemetleft poserons des défis inédits\guillemetright\ , au futur, témoingnent d'une part des effets qu'auront, à l'avenir, la numérisation de l'économie, les avancées technologiques, la mondialisation et les changements dans l'organisation du travail sur le monde du travail. D'autre part, ils mettent en évidence les problèmes qu'ils, par conséquent, pourraient constituer. 

Quels sont les principaux casse-têtes des politiques économiques liées au marché du travail ?\newline

Dès lors, dans une optique de connaissance approfondie de ce marché, il demeure important d'avoir une pleine vue du marché du travail actuel (I) et des principaux problèmes auxquels il fait déjà face (II). $$\star \star \star$$

\begin{center}
	{\bfseries Rédaction des paragraphes}	
\end{center}
\begin{enumerate}[label*=$\longrightarrow$]
	\item \textit{[ajouter argument ici]}
	\begin{itemize}
		\item \textbf{[ajouter argument ici]}
		\item \textbf{[ajouter argument ici]} \newline 
		\textbf{Transition :} [Ajouter texte ici] $$\star \star \star$$
	\end{itemize}
	\item \textit{[ajouter argument ici]}
	\begin{itemize}
		\item \textbf{[ajouter argument ici]}
		\item \textbf{[ajouter argument ici]} \newline 
		\textbf{Transition :} [Ajouter texte ici]
	\end{itemize}
\end{enumerate}
$$\star \star \star$$
\begin{center}
	\textbf{\underline{Conclusion}} 
\end{center} 

[Ajouter texte ici] $$\star \star \star$$
