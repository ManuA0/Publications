\newpage \begin{center}
	\subsubsection*{Innovation et croissance}\addcontentsline{toc}{subsection}{20240505 - UAO-M1-sujet (\'Eco-1)}
	%date - ISE 20... Sujet (\'Eco-1)
	%date - ISE 20... (Sujet 3)
\end{center}
$$\star \star \star$$
%\begin{center}
%	\textbf{\underline{Introduction}} 
%\end{center}
\indent Selon Nicholas Crafts et Terence Mills (2016), le PIB réel par tête anglais a subi des améliorations majeures à la suite de la révolution industrielle - période des grandes innovations. Passant de 0,2\% par an avant la révolution industrielle à 0,3\% à la suite de cette période pour atteindre les 1,25\% par an après cette période. Soit une variation positive du niveau de vie des habitants de l'Angleterre. Apparemment, la croissance semble emboîter le pas à l'innovation. Peut-on donc conclure qu'il existe un lien entre innovation et croissance ?

Selon l'INSEE, l’innovation désigne l’introduction sur le marché d'un produit ou d'un procédé nouveau ou significativement amélioré par rapport à ceux précédemment élaborés par une unité légale. On peut donc en déduire qu'il existe deux types d'innovation. Ce sont l'innovation de produits — biens ou services — et l'innovation de procédés, concernant les méthodes de production, de développement, de fabrication et de distribution. La croissance, en économie, désignerait l'augmentation de la production de biens et services, généralement mesurée grâce au PIB (produit intérieur brut), d'un pays au cours d'une période donnée. Ainsi, de manière relative, l'introduction d'un nouveau procédé ferait gagner l'économie en productivité et augmenterait par conséquent sa production : d'où la croissance. À son tour, la croissance entraînerait une hausse des profits des firmes. Cette hausse des profits permettrait donc aux firmes d'accroître leurs investissements en vue de financer la recherche — moteur de l'innovation. 

L'innovation et la croissance semblent s'auto-entretenir. Une économie, pour éviter la stagnation, a besoin d'innover de façon continuelle afin de stimuler sa croissance (Solow, 1954). Aussi, les bienfaits de la croissance semblent être à l'origine de la mise en place de mécanismes dans l'optique d'encourager l'innovation. Si pour les keynésiens, la politique de relance budgetaire est suffisante à créer une nouvelle croissance, les classiques eux la rechignent. Les économistes de l'école de l'école Schumpeterienne semblent plutôt miser sur l'innovation pour obtenir la croissance. Comment peut-on expliquer ce choix ? Peut-être s'agit-il du moyen le plus approprié au regard des limites des différentes politiques de croissance soutenable existantes. Faut-il en déduire que l'innovation génère une croissance de long terme au travers d'une boucle ? Cette hypothèse ne pourrait être crédible dans tous les cas. Existerait-il donc des cas où l'innovation et la croissance n'auraient aucun effet l'une sur l'autre ?

Est-ce que innovation et croissance s'auto-entretiennent-elles toujours  ?\\

Si l'innovation serait la cause de la croissance dans certains cas (I), la croissance pourrait, à long terme, être à son tour vecteur d'innovation (II), cependant l'une pourrait ne pas avoir d'effet sur l'autre et vis versa (III). 
$$\star \star \star$$

\begin{center}
	{\bfseries \underline{Plan détaillé}}	
\end{center}
\begin{enumerate}[label=\Alph*.]
	{\bfseries \item L'innovation serait de prime abord un vecteur de croissance.}
	\begin{enumerate}[label=\theenumi\arabic* -]
		\item {L'innovation entrainerait une hausse de la production}%\newline : Endogenous technological change
%		"Romer (1990) montre que l'innovation technologique, en augmentant le stock de connaissances et en améliorant l'efficacité productive, conduit à une augmentation de la production globale. L'innovation crée de nouvelles méthodes de production et de nouveaux produits, ce qui stimule la croissance économique de manière soutenue."
		%%%%%%%%%%%%%%%%%%%%%%%%%%%%%%%%%%%%%%%%%%%%%%%%%%%%%%%%%%%%%%%%%%%%%%%%%%%%%%%%%%%%%%%%%%%%%%%%%%%%%%%%
		\item {L'innvation serait aussi source d'accroissement du niveau de l'emploi}%\newline  : \underline{Capitalism, socialism and democracy}
%		Une référence classique pour soutenir cet argument est le livre "L'Économie de l'innovation" de Joseph Schumpeter. Schumpeter a introduit le concept de "destruction créatrice", expliquant comment les innovations technologiques conduisent à la disparition de certains emplois, mais également à la création de nouveaux emplois dans de nouveaux secteurs économiques.
		%%%%%%%%%%%%%%%%%%%%%%%%%%%%%%%%%%%%%%%%%%%%%%%%%%%%%%%%%%%%%%%%%%%%%%%%%%%%%%%%%%%%%%%%%%%%%%%%%%%%%%%%
		\item {L'innovation entrainerait une hausse de la demande.}%\newline  : \underline{The Theory of Economic Development: An Inquiry into Profits, Capital,}\newline \underline{Credit, Interest, and the Business Cycle}
%		Pour un argument plus directement lié à l'idée que l'innovation entraîne une hausse de la demande, un ouvrage plus approprié serait "The Theory of Economic Development" (1911) de Joseph Schumpeter. Schumpeter introduit le concept de "destruction créatrice", où les nouvelles innovations détruisent les anciennes structures économiques tout en créant de nouvelles opportunités et en stimulant la demande. Selon Schumpeter, les innovations provoquent des cycles économiques, augmentent la productivité, et créent de nouveaux produits et marchés, ce qui entraîne une augmentation de la demande.
		%%%%%%%%%%%%%%%%%%%%%%%%%%%%%%%%%%%%%%%%%%%%%%%%%%%%%%%%%%%%%%%%%%%%%%%%%%%%%%%%%%%%%%%%%%%%%%%%%%%%%%%%
	\end{enumerate}
	{\bfseries \item La croissance influencerait le niveau de l'innovation}
	\begin{enumerate}[label=\theenumi\arabic* -]
		\item {Certaines innovations n'entraineraient pas l'accroissement de la production.}%\newline EX : ordinanteur.
		%%%%%%%%%%%%%%%%%%%%%%%%%%%%%%%%%%%%%%%%%%%%%%%%%%%%%%%%%%%%%%%%%%%%%%%%%%%%%%%%%%%%%%%%%%%%%%%%%%%%%%%%
		\item {L'investissement entrainerait la croissance.}%\newline EX : Le cercle vertueux de Keynes. 
		%%%%%%%%%%%%%%%%%%%%%%%%%%%%%%%%%%%%%%%%%%%%%%%%%%%%%%%%%%%%%%%%%%%%%%%%%%%%%%%%%%%%%%%%%%%%%%%%%%%%%%%%
		\item {Des structures favorables permettraient la croissance.}%\newline EX : La politique structurelle. 
	\end{enumerate} 
	{\bfseries \item L'innovation et la croissance peuvent ne pas avoir d'effet l'une sur l'autre  contrairement à d'autres facteurs.}
	\begin{enumerate}[label=\theenumi\arabic* -]
		\item {Certaines innovations n'entraineraient pas l'accroissement de la production.}%\newline (R. \textsc{Solow}, 1987)
%		SOLOW exprime ainsi l’idée que les effets bénéfiques de l’informatique ne se lisent pas dans les statistiques de productivité, il s’appuie sur des analyses faisant apparaître une corrélation inverse entre les investissements informatiques et la productivité du travail aux Etats-Unis entre 1973 et 1995.
		%%%%%%%%%%%%%%%%%%%%%%%%%%%%%%%%%%%%%%%%%%%%%%%%%%%%%%%%%%%%%%%%%%%%%%%%%%%%%%%%%%%%%%%%%%%%%%%%%%%%%%%%
		\item {L'investissement favoriserait l'innovation ou la croissance.}%\newline (J.M \textsc{Keynes}, 1396) : Théorie générale de l'emploi, de la monnaie et de l'intérêt.
		%%%%%%%%%%%%%%%%%%%%%%%%%%%%%%%%%%%%%%%%%%%%%%%%%%%%%%%%%%%%%%%%%%%%%%%%%%%%%%%%%%%%%%%%%%%%%%%%%%%%%%%%
		\item {Des structures favorables permettraient la croissance.}%\newline EX : La politique structurelle. 
	\end{enumerate}
\end{enumerate}
$$\star \star \star$$
%\begin{center}
%	\textbf{\underline{Conclusion}} 
%\end{center} 

{[Ajouter texte ici pour la CONCLUSION]} $$\star \star \star$$



%%%%%%%%%%%%%%%%%%%%%%%%%%%%%%%%%%%%%%%%%%%%%%%%%%%%%%%%%%%%%%%%%%%%%%%%%%%%%%%%%%%%%%%%%%%%%%%%%%%%%%%%%%%%%%%%%%%%%%%%%%%%%%%%%%%%%%%%%%%%%%%%%%%%%%%%%%%%%%%%%%%%%%%%%%%%%%%%%%%%%%%%%%%%%%%%%%%%%%%%%%%%%%%%%%%%%%%%%%%%%%%%

\newpage \begin{center}
	\subsubsection*{Innovation et progrès technique}\addcontentsline{toc}{subsection}{20240505 - UAO-M1-sujet (\'Eco-2)}
	%date - ISE 20... Sujet (Eco-1)
	%date - ISE 20... (Sujet 3)
\end{center}
$$\star \star \star$$


%\noindent \textbf{Définition des termes du sujet :}
%
%\begin{enumerate}[label=\theenumi\arabic* -]
%	\item 
%\end{enumerate}

%\begin{center}
%	\textbf{\underline{Introduction}} 
%\end{center}

{Le progrès technique est l'ensemble des innovations concernant la nature des produits et les procédés de fabrication qui permettent la production, la diffusion des biens nouveaux ou de meilleure qualité, ou simplement, des gains de productivité dans la fabrication des produits déjà existants.}

La technique est l'ensemble des procédés destinés à la production.

l'innovation au sens étroit est conçue essentiellement de nature technique ou technologique, tandis qu'elle est conçue, au sens large, comme intégrant les changements dans l'organisation même de la production et des échanges.

Progrès technique et innovation sont-ils assimilables ?

Si le progrès et l'innovation paraissent de prime abord similaires (A), ils pourraient aussi être considérés comme deux concepts différents (B). Toutefois, ils semblent se chevaucher (C).$$\star \star \star$$

\begin{center}
	{\bfseries \underline{Plan détaillé}}	
\end{center}
\begin{enumerate}[label=\Alph*.]
	{\bfseries \item {Le progrès technique et l'innovation auraient des points en commun}}
	\begin{enumerate}[label=\theenumi\arabic* -]
		\item {L'innovation, tout comme le progrés technique, entrainerait une hausse de la productivité.}
		\item {L'innovation et le progrès technique seraient une source de diversification voire de renouvellement de l'offre.}  
		\item {L'innovation et le progrès technique seraient des moteurs de croissance.} 
		\item {L'innovation et le progrès technique seraient le fondement de l'amélioration du bien-être des population.} 
		\item {L'innovation et le progrès technique seraient tous deux endogènes à l'économie.}
%		{Transition :} [Ajouter texte ici]
	\end{enumerate}
	{\bfseries \item {L'innovation et le progrès technique pourraient aussi être considérés comme deux concepts différents.}}
	\begin{enumerate}[label=\theenumi\arabic* -]
		\item {Contrairement au progrès technique qui ne prend uniquement en compte les moyens et méthodes de production, l'innovation, elle, prendrait en compte l'organisation et l'amélioration des échanges.}
		\item {L'innovation serait un processus souvent discontinu et radical tandis que le progrès technique serait, quant à lui, un processus cumulatif et continu}
		\item {L'innovation se quantifierait en fonction de brevets tandis que le progrès technique serait évalué grâce la croissance de la productivité.}
		\item {L'innovation nécessiterait la mise en place de politique protection intellectuelle tandis que le progrès technique aurait besoin d'investissement dans l'éducation afin qu'ils soient soutenable.}  
%		{Transition :} [Ajouter texte ici]
	\end{enumerate}
	{\bfseries \item {Toutefois, l'innovation et le progrès technique se complèteraient partiellement.}}
	\begin{enumerate}[label=\theenumi\arabic* -]
		\item {Le progrès technique serait souvent le moteur de l'innovation.}
		\item {La progrès technique, au travers de son caractère amélioratif continuel, rendrait plus viables et efficaces les innovations existances.}
		\item {L'innovation serait le fil directeur du progrès technique en vue de l'améliorer.}
		\item {Le progrès technique fournirait des méthodes de diffusion de l'innovation} 
%		{Transition :} [Ajouter texte ici]
	\end{enumerate}
\end{enumerate}
$$\star \star \star$$
%\begin{center}
%	\textbf{\underline{Conclusion}} 
%\end{center} 

[Ajouter texte ici pour la CONCLUSION] $$\star \star \star$$

%%%%%%%%%%%%%%%%%%%%%%%%%%%%%%%%%%%%%%%%%%%%%%%%%%%%%%%%%%%%%%%%%%%%%%%%%%%%%%%%%%%%%%%%%%%%%%%%%%%%%%%%%%%%%%%%%%%%%%%%%%%%%%%%%%%%%%%%%%%%%%%%%%%%%%%%%%%%%%%%%%%%%%%%%%%%%%%%%%%%%%%%%%%%%%%%%%%%%%%%%%%%%%%%%%%%%%%%%%%%%%%%

\newpage \begin{center}
	\subsubsection*{Innovation et emploi}\addcontentsline{toc}{subsection}{20240505 - UAO-M1-sujet (\'Eco-2)}
	%date - ISE 20... Sujet (Eco-1)
	%date - ISE 20... (Sujet 3)
\end{center}
$$\star \star \star$$


%%\noindent \textbf{Définition des termes du sujet :}
%
%\begin{enumerate}[label=\theenumi\arabic* -]
%	\item 
%\end{enumerate}

%\begin{center}
%	\textbf{\underline{\textsf{Introduction}}} 
%\end{center}

{L'\textbf{innovation} est l'ensemble des changements dans l'organisation de la production et des échanges.}

{L'\textbf{emploi} se définit comme l'usage qu'on fait de quelque chose. Il désigne aussi l'occupation, la fonction d'une personne exerçant dans une organisation ou à son propre compte. Au niveau macroéconomique, l'emploi représente l'ensemble du travail fourni au sein d'une économie nationale, par l'ensemble de la population active qui n'est pas au chômage. L'emploi peut être dans le secteur public ou dans le secteur privé. Il peut aussi être subventionné par les pouvoir publics.}

Une \textbf{personne en emploi} au sens du bureau international du travail (BIT) est une personne de 15 ans ou plus ayant effectué au moins une heure de travail rémunéré au cours d'une semaine ou absente de son emploi sous certaines conditions de motifs (congés annuels, maladies, maternité etc.) et de durée.

{Le \textbf{marché de l'emploi} est un lieu, fictif ou réel, où se rencontrent offreurs et demandeurs d'emploi (celui qui offre sa force de travail.}

Quels sont les effets de l'innovation dur l'emploi ?\newline

Si les effets de l'innovation sur l'emploi peuvent être mesurables (A), d'autres par contre semblent être qualitatifs (B). $$\star \star \star$$

\begin{center}
	{\bfseries \underline{Plan détaillé}}	
\end{center}
\begin{enumerate}[label=\Alph*.]
	{\bfseries \item {L'innovation semble avoir des effets quantitatifs sur l'emploi.}}
	\begin{enumerate}[label=\theenumi\arabic* -]
		\item \label{A.1\_UAO\_M1\_S3} {L'innovation apparaît comme destructeur d'emplois à court terme.}
		\item {Sur le long terme, l'innovation semble être à l'origine de la création de nouveaux emplois.} 
		\item {L'innovation accroîtrait les inégalités salariales.}
		\item {L'innovation entrainerait des gains de productivité dans le travail de l'employé.} 
%		\textbf{Transition :} [Ajouter texte ici] $$\star \star \star$$
	\end{enumerate}
	{\bfseries \item {L'innovation paraît aussi avoir des effets non mesurables sur l'emploi.}}
	\begin{enumerate}[label=\theenumi\arabic* -]
		\item {L'innovation semble être, de prime abord, la cause de la tertiarisation des emplois.}
		\item {L'innovation créerait de même des emplois de plus en plus qualifiés au détriment des autres.}  
		\item {L'innovation serait par conséquent à l'origine de la précarisation de l'emploi.}
		\item {L'innovation favoriserait l'amélioration des conditions de travail.}
%		\textbf{Transition :} [Ajouter texte ici]
	\end{enumerate}
%	{\bfseries \item {[ajouter argument ici]}}
%	\begin{enumerate}[label=\theenumi\arabic* -]
%		\item {[ajouter argument ici]}
%		\item {[ajouter argument ici]}  
%%		\textbf{Transition :} [Ajouter texte ici]
%	\end{enumerate}
\end{enumerate}
$$\star \star \star$$
%\begin{center}
%	\textbf{\underline{Conclusion}} 
%\end{center} 

[Ajouter texte ici pour la CONCLUSION] $$\star \star \star$$

%%%%%%%%%%%%%%%%%%%%%%%%%%%%%%%%%%%%%%%%%%%%%%%%%%%%%%%%%%%%%%%%%%%%%%%%%%%%%%%%%%%%%%%%%%%%%%%%%%%%%%%%%%%%%%%%%%%%%%%%%%%%%%%%%%%%%%%%%%%%%%%%%%%%%%%%%%%%%%%%%%%%%%%%%%%%%%%%%%%%%%%%%%%%%%%%%%%%%%%%%%%%%%%%%%%%%%%%%%%%%%%%
