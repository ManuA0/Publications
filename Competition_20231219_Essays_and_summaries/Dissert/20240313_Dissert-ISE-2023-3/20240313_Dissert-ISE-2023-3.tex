\newpage \begin{center}
	\subsubsection*{« Il faut cultiver notre jardin. » Telle est la conclusion du conte philosophique, Candide, de Voltaire, écrivain français du XVIIIème siècle. \newline Le refus du monde extérieur est-il une caractéristique de nos sociétés aujourd’hui ?}\addcontentsline{toc}{subsection}{20230313 - ISE 2023 (Sujet 3)}
	%date - ISE 20... Sujet (Eco-1)
	%date - ISE 20... (Sujet 3)
	$\star \star \star$
\end{center}

%\noindent \textbf{Définition des termes du sujet :}
%
%\begin{itemize}
%	\item 
%\end{itemize}

\begin{center}
	\textbf{\underline{Introduction}} 
\end{center}

[ajouter texte ici] \newline

Se demander si le refus du monde extérieur pourrait constituer une caractéristique de nos sociétés comtemporaines revient, en quelque sorte, à se demander si l'isolement serait devenue la tendance actuelle dans nos sociétés. Dit autrement, les sociétés auraient-elles toutes en commun l'isolement comme style de vie ? Serait-il le seul élément qu'elles auraient en commun ? Dès lors ;

L'isolement peut-il être considéré comme la seule chose que partagent toutes les sociétés ?\newline

Bien que l'isolement fasse partie du quotidien de nos sociétés (I), elle ne pourrait cependant être le seul éléments qu'elles partagent toutes. (II).$$\star \star \star$$

\begin{center}
	\textbf{\underline{Rédaction des paragraphes}} 
\end{center}

\begin{enumerate}[label*=$\longrightarrow$]
	\item \textit{L'isolement serait le style de vie adopté par l'ensemble des sociétés actuelles.}
	\begin{itemize}
		\item \textbf{Dans nos sociétés, les gens ont tendance à s'isoler physiquement du fait de leur attrait au monde virtuel.} En effet, l'avènement et le développement des réseaux sociaux ont détruit les frontières physiques qui existaient entre nous au point où tout le monde peut avoir des amis à l'extérieur. Cependant, ces nouveaux liens qu'il peut dès lors tisser entretient chez certains un manque de désir d'avoir des relations physiques avec les gens qui les entoure ou même d'entretenir celles qu'ils ont déjà, et les pousse ainsi à s'isoler physiquement du monde. Cela s'illustre bien par le cas des jeunes qui, de nos jours, passent de longues heures sur les réseaux sociaux négligeant par ainsi les interactions physiques avec ses amis et sa famille.
		\item ... \textbf{Ils ont encore tendance à s'isoler afin de se protéger de certaines situations.} Au regard des vices dans lesquels certaines relations ont entrainé certaines personnes, d'aucuns trouvent préférable de s'isoler. Car s'isoler est le seul moyen pour eux d'échapper à l'influence des autres. Dès lors, être seul devient primordial pour ces personnes car elles se sentent plus en sécurité avec elles-mêmes qu'en compagnie des autres. c'est en cela que l'adage \guillemetleft\ mieux vaut être seul, que mal accompagné\ \guillemetright\ vient soutenir cette vision de vie que nombreux sont les personnes dans nos sociétés ont adoptée. \newline \textbf{Transition :} Si l'isolement dû à la préférence du monde virtuel au monde réel et au désir de se protéger constitue une caractéristique de nos sociétés d'aujourd'hui, il ne pourrait néanmoins être le seul point commun de toutes les sociétés.$$\star \star \star$$
	\end{itemize}
	\item ... \textit{Par ailleurs, les sociétés contemporaines auraient toutes d'autres éléments en commun.}
	\begin{itemize}
		\item \textbf{Le besoin serait une caractéristique que toutes les sociétés partagent.} D'une part, c'est le besoin de se nourrir qui a poussé les sociétés primitives à chasser. D'autre part, c'est besoin d'avoir plus de libertés qui a poussé et qui continue d'inspirer nos sociétés contemporaines à se battre pour leurs droits. Dit autrement, c'est le besoin qui entraine des transformations dans nos sociétés. Or nos sociétés étant toutes en évolution, par conséquent, le besoin fait partie de notre vécu. Dès lors, toutes les sociétés contemporaines partagent aussi le besoin.
		\item ...\textbf{La culture serait aussi une caractéristique que les sociétés contemporaines partagent.} La culture est un élément essentiel de toute société car elle façonne l'identité collective, les interactions sociales et les valeurs partagées. Elle constitue donc une part importante du vécu de tout homme. En effet, elle est à la fois sa création et sa source d'inspiration. C'est à juste titre qu'Antoine de Rivarol affirme que \guillemetleft\ l'homme sans culture est un arbre sans fruit\ \guillemetright\ dans \underline{Pensées inédites de Rivarol}. Cela pour souligner que l'absence de culture chez l'homme le rend improductif. Or, puisque aucune société n'a jamais manqué de créativité au cours de l'histoire, on peut sans aucun doute dire que toutes les sociétés ont une culture. \newline \textbf{Transition :} Outre l'isolement, le besoin et la culture sont donc des éléments que toutes les sociétés contemporaines partagent." 
	\end{itemize}
\end{enumerate}
$$\star \star \star$$
\begin{center}
	\textbf{\underline{Conclusion}} 
\end{center}

En somme l'isolement fait partie de notre vécu à tous du fait de notre fort attrait au monde virtuel et de notre volonté de nous protéger de certaines situations. Toutefois, il ne peut être considéré comme le seul élément que nous tous partageons. Parceque, le besoin et la culture sont aussi deux éléments que nous tous avons en commun.$$\star \star \star$$