\begin{center}
	\subsubsection*{{Hommages et commémorations vous semblent-elles des cérémonies toujours nécessaires ?}}\addcontentsline{toc}{subsection}{20230921- ISE 2023 (sujet 2)}
	$\star \star \star$
\end{center}

\noindent \textbf{Définition des termes du sujet :}

\begin{itemize}
	\item \textbf{Rendre hommage à quelqu’un} : Témoigner son respect, son admiration, sa reconnaissance en-vers quelqu’un.
	\item \textbf{Commémorer} : Rappeler par une cérémonie le souvenir d’une personne ou d’un événement
\end{itemize}

\begin{center}
	\textbf{\underline{Introduction}} 
\end{center}


\hspace*{0.5cm}Les hommages et les commémorations ont joué un grand rôle dans l’histoire de l'humanité dans le sens qu’elles permettent de ne pas oublier certains évènements et personnes importantes. Ces cérémonies, souvent empreintes d'émotion et de solennité, revêtent donc une signification particulière pour la société. Toutefois, dans un monde en constante évolution, se pose la question de leur pertinence et de leur nécessité.\\
\hspace*{0.5cm}Les hommages sont des gestes ou des actes visant à rendre hommage à une personne ou à un événement, souvent par le biais de cérémonies publiques ou privées. \textit{« Rendre hommage à quelqu’un »} serait donc témoigner son respect, son admiration, sa reconnaissance en-vers quelqu’un. Les commémorations sont quant à elles des cérémonies destinées à rappeler et à honorer un événement, une personne ou un groupe de personnes à travers des rites spécifiques.\\
\hspace*{0.5cm}La récente absence d’hommage de la part de Vladimir Poutine envers Nikita Khrouchtchev soulèverait la question fondamentale de l’importance des hommages et des commémorations dans notre société actuelle. Cette omission met en lumière les tensions politiques et les différences idéologiques qui peuvent exister entre les dirigeants, mais soulèverait également des interrogations plus profondes sur la nécessité de célébrer le passé, de commémorer des personnalités historiques et de perpétuer la mémoire collective.\\
\hspace*{0.5cm}Comment pourrait-on évaluer la pertinence des hommages et des commémorations à cette ère ?\\
\hspace*{0.5cm}Bien que les hommages et les commémorations seraient jugés par certains comme utiles, d'aucuns remettraient en cause leur existence. $$\star \star \star$$

%\begin{center}
%	{\bfseries \underline{Rédaction des paragraphes}}	
%\end{center}
%\begin{enumerate}[label*=$\longrightarrow$]
%	\item \textit{[ajouter argument ici]}
%	\begin{itemize}
%		\item \textbf{[ajouter argument ici]}
%		\item \textbf{[ajouter argument ici]} \newline 
%		\textbf{Transition :} [Ajouter texte ici] $$\star \star \star$$
%	\end{itemize}
%	\item \textit{[ajouter argument ici]}
%	\begin{itemize}
%		\item \textbf{[ajouter argument ici]}
%		\item \textbf{[ajouter argument ici]} \newline 
%		\textbf{Transition :} [Ajouter texte ici]
%	\end{itemize}
%\end{enumerate}
%$$\star \star \star$$
%\begin{center}
%	\textbf{\underline{Conclusion}} 
%\end{center} 
%
%[Ajouter texte ici] $$\star \star \star$$

\hspace*{0.5cm}Les hommages et commémorations seraient, de prime abord, utiles à cause de leurs fonctions historiques et sociétales pour notre société.\\
\hspace*{0.5cm}En effet, ils permettraient de préserver la mémoire collective, car permettant la transmission de l’histoire entre les générations d’une part, ils renforceraient, d’autre part, les liens entre les générations. Parce que permettant aux différentes générations de connecter à des évènements et des figures historiques, ils offrent l’occasion d’apprendre et de transmettre l’histoire, les valeurs et les leçons du passé aux jeunes générations ; créant ainsi un dialogue intergénérationnel. C’est le cas de ces jeunes français qui ont pu renouer avec l’histoire de leurs vaillants aïeux et en apprendre davantage grâce à la commémoration en 2018 du centenaire de la grande guerre ; comme le souligne Alexandre LAFFON dans « géopolitiques de la commémoration du Nationalisme » de 2020.\\
\hspace*{0.5cm}Les hommages et les commémorations joueraient aussi un rôle essentiel dans la préservation et la valorisation de la culture d’une société. En tant que véritables témoins du passé, ils permettraient de préserver les coutumes, les traditions et les évènements marquants qui ont façonnés la culture d’une société. En mettant en avant des moments culturellement significatifs, ils permettraient à une société de se reconnecter avec son patrimoine culturel et d’en tirer fierté. En outre, ces célébrations contribueraient à créer un sentiment d’identité culturelle partagée, renforçant ainsi le tissu social et culturel ; telle la commémoration de la guerre de la sécession le 19 juin afin de préserver la mémoire de cet évènement crucial et de valoriser la diversité culturelle américaine. \\
\hspace*{0.5cm}Nonobstant les vertus historiques et sociétales qui les sont rattachés, il demeure cependant qu’ils soient critiqués dans certains cas.
$$\star \star \star$$
\hspace*{0.5cm}Dans certains cas, les hommages et les commémorations seraient néanmoins remis en question.\\
\hspace*{0.5cm}Tout d’abord, les valeurs et priorités de la société moderne susciteraient des interrogations sur l’utilité des hommages et des commémorations. En effet, dans un monde où l’efficacité des ressources est primordiale, certains remettent en question la pertinence de ces célébrations couteuses en termes de temps et de ressources. De plus, les priorités axées sur la lutte contre l’injustice sociale et l’inégalité questionneraient l’accent mis sur des commémorations plutôt que sur des actions contemporaines en faveur du changement. C’est donc ce qu’a fait le Mali en décidant de célébrant modestement sa 63ème fête d’indépendance afin d’allouer une grande partie de ce budget aux familles victimes d’une attaque terroriste durant la veille. Par ailleurs, les valeurs contemporaines d’inclusivité et de diversité ont également un impact sur cette remise en question car elles amènent à réévaluer les figures historiques à la lumière des critères moraux et sociaux culturels.\\
\hspace*{0.5cm}En sus, ces célébrations conçus pour préserver la mémoire collective et promouvoir la culture, seraient parfois détournées à des fins commerciales et politiques. Du fait que certaines entreprises exploitent ces évènements pour des gains économiques en utilisant des symboles, images…liés à l’événement commémoré, parfois sans respecter l’authenticité culturelle ou historique. Ou même que les dirigeants politiques pourraient politiser ces célébrations pour renforcer leur image et leur pouvoir, tandis que des groupes extrémistes pourraient les utiliser pour promouvoir leur doctrine radicale. Dès lors, nous pouvons citer à titre d’exemple la manipulation de la fête de la fondation de la république de Corée du Nord. Car, le régime nord-coréen utiliserait ces festivités du 9 septembre afin promouvoir de son idéologie et sa suprématie militaire à l’aide de défilés militaires massifs et spectacles grandioses organisés afin de renforcer le culte de la personnalité des dirigeants. \\
\hspace*{0.5cm}Par conséquent, les valeurs, priorités et abus à l’égard de ces cérémonies dans notre société moderne actuelle remettraient en cause leur pertinence. 
$$\star \star \star$$
\hspace*{0.5cm}Pour conclure, bien que hommages et commémoration permettraient de préserver la mémoire collective et de valoriser les cultures d’une société, ce qui les rendraient nécessaires ; leur utilisation au service d’intérêt politiques ou commerciales et les nombreux défis économiques du contexte actuel les rendraient cependant moins pertinents, voir inutiles dans certains cas. \newpage