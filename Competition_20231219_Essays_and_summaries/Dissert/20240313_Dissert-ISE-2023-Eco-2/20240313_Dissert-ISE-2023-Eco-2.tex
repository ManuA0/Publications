\newpage \begin{center}
	\subsubsection*{Inégalités économiques, sociales et territoriales : Après avoir décrit les facteurs déterminants et les principales dynamiques au plan mondial, vous analyserez la situation dans un ou deux pays.}\addcontentsline{toc}{subsection}{20230313 - ISE 2023 Sujet (Eco-2)}
	%date - ISE 20... Sujet (Eco-1)
	%date - ISE 20... (Sujet 3)
\end{center}
$$\star \star \star$$

\noindent \textbf{Définition des termes du sujet :}

\begin{itemize}
	\item Les \textbf{inégalités économiques} font allusion aux disparités liées au partage inégal des richesses créées,
	\item Les \textbf{inégalités sociales} s'apparentent aux problèmes liés aux genres, en plus d'autres facteurs touchant à l'homme,
	\item les \textbf{inégalités territoriales} désignent l'ensemble des disparités existantes entre différentes régions, villes, voire des quartiers en terme d'opprotunités économiques (quatiers huppés et zones précaires), social, saniatires etc.
\end{itemize}

\begin{center}
	\textbf{\underline{Introduction}} 
\end{center}

Sachant que les inégalités sociales sont en majeure partie liées à une forte disparité entre les revenus (10\% de la population détient 90\% des richesses) outre d'autres facteurs tels que la couleur de peau etc. Ceux-ci s'accompagne toutefois d'une inégale disponibilité d'opportunités. Par conséquent, il demeure important de se pencher sur la relation auto-entretenue entre les inégalités économiques, sociales et territoriales.

Quelle analyse synoptique des inégalités économiques, sociales et territoriales peut être envisageable ? \newline

Bien que certains facteurs favorisent et participent à la dynamique de ces inégalités à l'échelle mondiale (I), il serait néanmoins nécessaire de se pencher sur la question dans le cas de certains pays (II). $$\star \star \star$$

\begin{center}
	{\bfseries \underline{Rédaction des paragraphes}}	
\end{center}
\begin{enumerate}[label*=$\longrightarrow$]
	\item \textit{[ajouter argument ici]}
	\begin{itemize}
		\item \textbf{[ajouter argument ici]}
		\item \textbf{[ajouter argument ici]} \newline 
		\textbf{Transition :} [Ajouter texte ici] $$\star \star \star$$
	\end{itemize}
	\item \textit{[ajouter argument ici]}
	\begin{itemize}
		\item \textbf{[ajouter argument ici]}
		\item \textbf{[ajouter argument ici]} \newline 
		\textbf{Transition :} [Ajouter texte ici]
	\end{itemize}
\end{enumerate}
$$\star \star \star$$
\begin{center}
	\textbf{\underline{Conclusion}} 
\end{center} 

[Ajouter texte ici] $$\star \star \star$$