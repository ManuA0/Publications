\newpage %\begin{center}
	\subsubsection*{Dans un document intitulé \guillemetleft \textit{Les infrastructures à l’horizon 2030}\guillemetright,  l’OCDE note que : \newline \guillemetleft Les réseaux d’infrastructure jouent un rôle vital dans le développement économique et social. De plus en plus interdépendants, ils constituent un moyen d’assurer la fourniture et la prestation de biens et de services qui concourent à la prospérité et à la croissance économique et contribuent à la qualité de vie. La demande d’infrastructure est appelée à sensiblement augmenter dans les décennies à venir, sous l’impulsion de facteurs majeurs de changement comme la croissance économique mondiale, le progrès technologique, le changement climatique, l’urbanisation et l’intensification de la congestion. Toutefois, les défis à relever sont multiples (…)\guillemetright \newline \newline Après avoir rappelé les théories relatives à la croissance, vous préciserez le rôle des infrastructures et les défis auxquels sont confrontées les nations à différents niveaux de croissance pour en assurer la pérennité et le développement. }\addcontentsline{toc}{subsection}{20240228 - ISE 2018 Sujet (Eco-1)}
%\end{center}
$$\star \star \star$$

\begin{center}
	\textbf{\underline{Introduction}} 
\end{center}

Dans ce sujet, le mot qui attire plus notre attention est \guillemetleft vital\guillemetright. Considérer que les réseaux d'infrastructure jouent un rôle vital dans le développement économique revient en fait à mettre en évidence qu'il ne peut avoir de développement sans infrastructures. Dès lors, les infrastructures occupent une place centrale dans le processus développement. Par ailleurs, sachant que ce processus débute par la croissance économique, il est par conséquent primordial de se demander :

Dans quelle mesure les infrastructures sont-ils un des facteurs déterminants de la croissance ?\\

Pour répondre à cette quesion, il serait important, premièrement, de rappeler les théories relatives à la croissance (I) puis en second lieu, de montrer en quoi les infrastuctures peuvent favoriser la croissance tout en évoquant les défis liés au développement (II). $$\star \star \star$$

\begin{center}
	{\bfseries Rédaction des paragraphes}	
\end{center}
\begin{enumerate}[label*=$\longrightarrow$]
	\item \textit{La croissance, de façon empirique, est la résultante de l'investissement,}
	\begin{itemize}
		\item \textbf{Lorsque les dépenses de l'état augmentent alors la production s'accroit d'où la croissance.} (Le multiplicateur d'investissement de Keynes.)
		\item \textbf{L'investissement dans le capital humain favorise la croissance} (Le modèle de Lucas) \newline 
		\textbf{Transition :} [Ajouter texte ici] $$\star \star \star$$
	\end{itemize}
	\item \textit{[ajouter argument ici]}
	\begin{itemize}
		\item \textbf{[ajouter argument ici]}
		\item \textbf{[ajouter argument ici]} \newline 
		\textbf{Transition :} [Ajouter texte ici]
	\end{itemize}
\end{enumerate}
$$\star \star \star$$
\begin{center}
	\textbf{\underline{Conclusion}} 
\end{center} 

[Ajouter texte ici] $$\star \star \star$$