\subsection*{\underline{\hyperref[item-UAO-M1]{Paragraphe n°1 :}}}\addcontentsline{toc}{subsection}{Paragraphe n°1}

\noindent \textbf{L'innovation serait à l'origine de l'amélioration des moyens de production.} \'A partir des découvertes faites par la recherche, l'innovation serait la mise en application de ces découvertes, par les firmes, en vue de réorganiser de manière partielle ou totale leur processus de fonctionnement. Cela, dans le but d'être plus efficient. Ainsi, si l'innovation permet d'atteindre un processus de fonctionnement, dans le cas de la production, elle pourrait donc être considérée comme le moyen permettant de rendre les moyens de production plus efficace. Ainsi, pourrons-nous citer, à titre d'exemple, la machine à vapeur de James \textsc{Watt} \& Matthew \textsc{Boulton} comme l'une des innovations majeures faites en 1778 et ayant permis l'amélioration des moyens de production dans de nombreux secteurs tels que le secteur des mines, du coton etc. En effet, cette machine a joué un grand rôle dans l'amélioration de bon nombre de méthodes voire moyens de production. En conséquence, l'innovation est à l'origine de l'amélioration des moyens de production.

%%%%%%%%%%%%%%%%%%%%%%%%%%%%%%%%%%%%%%%%%%%%%%%%%%%%%%%%%%%%%%%%%%%%%%%%%%%%%%%%%%%%%%%%%%%%%%%%%%%%%%%%%%%%%%%%%%%
\subsection*{\underline{Paragraphe n°i :}}\addcontentsline{toc}{subsection}{Paragraphe n°i}

\noindent \textbf{[Ajouter argument ici]}