\documentclass[a4paper, 12pt, openany]{book}
\renewcommand{\contentsname}{Table des matières}
\usepackage{polyglossia} %to set a language
\setmainlanguage{french}
\setotherlanguage{english}
\usepackage{hyperref}
\hypersetup{
	colorlinks=true,
	linkcolor=black,
	urlcolor=blue,
	citecolor=black,    
	pdftitle={},
	pdfauthor={Emmanuel ASSOHOU},
	pdfsubject={},
	pdfkeywords={},
}
\usepackage{xcolor} 
\usepackage{fontspec} %to specify a font
\usepackage{unicode-math} %to use all the math operators and symbols
\setmainfont{Times New Roman}
\setmathfont{Latin Modern Math}
\usepackage{enumitem} %to provide enhanced control over lists, allowing to customize the appearance and formatting of various types of lists
\usepackage{graphicx} %to use pictures in the document
\usepackage{setspace}
\setstretch{1.5}
\usepackage{tikzpagenodes} %provides convenient ways to work with page related coordinates and dimensions in TikZ
\usepackage[margin=2cm]{geometry}
\usepackage{ragged2e} %to provide enhanced justification commands for text alignment
\usepackage{amsmath} %ptovides advanced mathematical typesetting features and environments 
\usepackage{float}
\usepackage{emoji}
\usepackage{mfirstuc} %for capitalizing
\usepackage{stmaryrd}

\usepackage{tikz}
\usepackage{pgfplots}
\pgfplotsset{compat=1.15}
\pgfplotsset{soldot/.style={only marks,mark=*, line width=0.5pt, mark size=2pt}}
\pgfplotsset{holdot/.style={fill=white,only marks,mark=*, line width=1pt, mark size=2pt}}
\renewcommand{\contentsname}{Table des matières}
\usepackage{polyglossia} %to set a language
\setmainlanguage{french}
\setotherlanguage{english}
\usepackage{hyperref}
\hypersetup{
	colorlinks=true,
	linkcolor=black,
	urlcolor=blue,
	citecolor=black,    
	pdftitle={},
	pdfauthor={Emmanuel ASSOHOU},
	pdfsubject={},
	pdfkeywords={},
}
\usepackage{xcolor} 
\usepackage{amsmath}
\usepackage{longtable}
\usepackage{listings} % Include the listings package



\begin{document}

\begin{titlepage} % Suppresses displaying the page number on the title page and the subsequent page counts as page 1
	\newgeometry{margin=1cm}
	\begin{tikzpicture}[remember picture, overlay]
		\draw[line width=1.6pt](current page text area.south west) rectangle(current page text area.north east);
	\end{tikzpicture}
	\newcommand{\HRule}{\rule{\linewidth}{0.5mm}} % Defines a new command for horizontal lines, change thickness here
	
	%\center % Centre everything on the page
	
	%------------------------------------------------
	%	Headings
	%------------------------------------------------
	\centering
	\textsc{\large MINISTÈRE DE L’ENSEIGNEMENT SUPÉRIEUR ET DE LA RECHERCHE SCIENTIFIQUE DE CÔTE D'IVOIRE}\\%[1.5cm] % Main heading such as the name of your university/college
	%\begin{figure}[H]
		\centering
		\includegraphics[width=.3\textwidth]{C:/Users/Emmanuel/Documents/latex-files/Photos/others/download.png}
	%\end{figure}
	\textsc{\large \\\textbf{UFR} : SCIENCES ÉCONOMIQUES ET DÉVELOPPEMENT}\\[0.5cm] % Major heading such as UFR name
	\textsc{\large \textbf{Option} : Économie}\\[0.5cm] % Major heading such as course name
	
	 % Minor heading such as course title
	
	%------------------------------------------------
	%	Title
	%------------------------------------------------
	
	\HRule\\[0.4cm]
	\fontsize{20}{1.5}{ \bfseries DEVOIR DE LOGICIELS D'ÉCONOMÉTRIE}\\[0.4cm] % Title of your document
	
	\HRule\\[1cm]
	
	%------------------------------------------------
	%	Author(s)
	%------------------------------------------------
	
	\begin{minipage}{0.4\textwidth}
		\begin{flushleft}
			\small
			\textit{\textbf{Étudiant}}\\
			\textsc{Assohou} Tano Franck Emmanuel 
			(\textbf{CI0321003010}) % Your name
			 
		\end{flushleft}
	\end{minipage}
	~
	\begin{minipage}{.4\textwidth}
		\begin{flushright}
			\small
			\textit{\textbf{Enseignant}}\\
			 \textsc{Dr NAHOUSSÉ} % Supervisor's name
		\end{flushright}
	\end{minipage}
	
	% If you don't want a supervisor, uncomment the two lines below and comment the code above
	%{\large\textit{Author}}\\
	%John \textsc{Smith} % Your name
	
	%------------------------------------------------
	%	Date
	%------------------------------------------------
	
	\vfill\vfill\vfill % Position the date 3/4 down the remaining page
	
	{\large 2022 - 2023} % Date, change the \today to a set date if you want to be precise
	
	%------------------------------------------------
	%	Logo
	%------------------------------------------------
	
	%\vfill\vfill
	%\includegraphics[width=0.2\textwidth]{}\\[1cm] % Include a department/university logo - this will require the graphicx package
	 
	%----------------------------------------------------------------------------------------
	
	\vfill % Push the date up 1/4 of the remaining page
	
\end{titlepage}

%\maketitle{}
\tableofcontents

\newpage
En utilisant les données du tableau 1,  estimons le modèle de régression linéaire multiple suivant 
$Conso\_pub_t=alpha+\beta Force\_travail_t+\delta IDE_t+\gamma inflation_t+\varepsilon_t$,  tels que $Conso\_pub$ désigne les dépenses de consommation publique ;  $Force\_travail$ : le taux d’emploi ;IDE : les investissements directs étrangers et inflation : l’indice des prix à la consommation, grâce au logiciel STATA 16. Puis, nous interprèterons les résultats après avoir effectué les tests post estimation.

\section{Existence de corrélation entre les différentes variables}
\vspace*{-.5cm}
\begin{figure}[H]
    \centering
    \includegraphics[width=.65\textwidth]{C:/Users/Emmanuel/Documents/latex-files/Photos/others/corrélation.jpg}
    \caption{Tableau de corrélation}
    \label{}
\end{figure}
Le tableau ci-dessus a été obtenu en utilisant la méthode \colorbox{black}{\textcolor{white}{\texttt{correlate}}}.
En effet, il existe une corrélation négative entre la plupart des variables exceptées entre la variable $Force\_travail$ et les variables $conso\_pub$ et $inflation$, qui elles sont liées de manière positive. Ainsi, tout comme le taux d'emploi et les dépenses de consommation publique, le taux de l'emploi et l'inflation varient dans le même sens.

\vspace*{-0.8cm}
\section{Régression linéaire multiple}

\begin{figure}[H]
    \centering
    \includegraphics[width=.65\textwidth]{C:/Users/Emmanuel/Documents/latex-files/Photos/others/regression.jpg}
    \caption{Tableau de regression multiple linéaire}
    \label{}
    
\end{figure}

En utilisant la méthode \colorbox{black}{\textcolor{white}{\texttt{regress}}} on obtient le tableau ci-dessus, donnant les différentes valeurs des coefficients du modèle et des informations supplémentaires relatives au modèle. Cela nous permet donc d'écrire que la régression multiple linéaire de exercice est :
\vspace*{-.5cm}
\begin{multline*}
	Conso\_pub_t= -1.136147\ +\ 0.2566749\ Force\_travail_t\ ‐ \ 0.5691375\ IDE_t\ \\‐ \ 0.1695814\ inflation_t\ +\ \hat{\varepsilon}_t
\end{multline*}
\vspace*{-1.5cm}
\begin{spacing}{1.4}
À ce niveau nous ne pouvons pas interpréter le modèle. Toutefois, connaissant le coefficient de régression multiple $R^2$ , il nous est possible d'affirmer que les variables $force\_travail$, $IDE$ et $inflation$ expliquent à 48.72\% notre modèle.
\end{spacing}
\vspace*{-0.5cm}
\section{Normalité des erreurs}
\begin{figure}[H]
    \centering
    \includegraphics[width=.65\textwidth]{C:/Users/Emmanuel/Documents/latex-files/Photos/others/Sktest residuals.jpg}
    \caption{Tableau du test de normalité des résidus}
\end{figure}
Pour cette tâche, après avoir prédit les erreurs du modèle grâce à
la méthode \colorbox{black}{\textcolor{white}{\texttt{predict}}}, nous avons utilisé l'un des tests de normalité donné par la fonction \colorbox{black}{\textcolor{white}{\texttt{sktest}}}. Ce qui nous a permis d'avoir le ableau ci-dessus, résumant les résultats du test.

Ainsi, on a $p-value=0.0349\%$.
\begin{itemize} [label=$\bullet$]
    \item \textbf{Comparons \textit{p-value} à 5\%.} 
\end{itemize}

Au risque de rejet de l'hypothèse nulle de normalité des erreurs, nous pouvons constater que
$p-value<5\%$. Par conséquent, les erreurs du modèle ne suivent pas une loi normale car en plus de 
la comparaison faite plus haut, probabilité de se tromper est de 3.49\%.

%\vspace*{-.8cm}
\section{Les tests de significativité}
\subsection{Test de significativité globale}
\begin{figure}[H]
    \centering
    \includegraphics[width=.65\textwidth]{C:/Users/Emmanuel/Documents/latex-files/Photos/others/Test de significativité globale.jpg}
    \caption{Tableau du test de significativité globale des variables du modèle}
\end{figure}

\noindent Ce résultat a été obtenu grâce à la fonction \colorbox{black}{\textcolor{white}{\texttt{test}}} de 
STATA 16. Sachant que nous avons $F(3, 25)= 7.92$, faisons donc une comparaison avec le quartiles 
d'ordre 0.95. Pour un test à 5\% à savoir ${F_{0.95}(3, 25)$ $={2.99124091}}$
 Nous constatons que : $F(3, 25)>F_{0.95}(3, 25)$\footnote{INVERSE.LOI.F.DROITE(0.05;3;25) dans Excel.}, donc le modèle est globalement significatif. 
Il s'en suit que les variables $Force\_travail$, $inflation$ et $IDE$ apportent de l'information
nécessaire à la compréhension des dépenses de consommation publique.

\subsection{Test de significativité individuelle}
\begin{figure}[H]
    \centering
    \includegraphics[width=.65\textwidth]{C:/Users/Emmanuel/Documents/latex-files/Photos/others/Stata output2_Page_2 (2).jpg}
    \caption{Tableau des tests de significativité individuelle}
\end{figure}

On peut remarquer que le tableau obtenu grâce à la méthode \colorbox{black}{\textcolor{white}{\texttt{ttest}}}
nous donne des informations nécessaires pour tester l'hypothèse de significativité 
des variables $Force\_travail, IDE$ et \textit{inflation}.
Sous $H_0$ : $t_{\hat{Force\_travail}}(28)=96.6329$. \newpage \noindent Au risque de 5\%, $t_{0.975}(28) = 2.048407142\footnote{LOI.STUDENT.INVERSE(0.05;28) dans Excel.}$, 
nous constatons que le coefficient $\beta$ associé à ($Force\_travail_t$ — \ \textit{taux de l'emploi}) 
est significatif
puisque:  $$\left|t_{{\hat{Force\_travail_t}}(28)}\right|>{t_{0.975}}(28)=2.048407142$$
De même par analogie, on a sous $H_0$ :
\begin{equation*}
    \begin{cases}
        t_{\hat{IDE}}(28)=8.4987 \\ 
        t_{\hat{inflation}}(28)=3.7403
    \end{cases}
\end{equation*}

Par conséquent, au risque de 5\%, les variables $IDE$ et $inflation$ sont toutes deux 
individuellement significatives car :
\begin{equation*}
    \begin{cases}
        \left|t_{\hat{IDE}}\right| > t_{0.975} \\ 
        \left|t_{\hat{inflation}}\right| > t_{0.975}
    \end{cases}
\end{equation*}

En somme, toutes les variables du modèle semblent contribuer à apporter de l'information significative
de manière individuelle dans l'explication de la variable $conso\_publique$.

\section{Hétéroscédasticité}

\hspace*{.8cm}Utilisons la fonction \colorbox{black}{\textcolor{white}{\texttt{httest}}} pour générer le \texttt{«test Breusch‐Pagan»}
afin de savoir s'il existe des variables dans notre modèle qui évoluent plus vite que les autres
dans l'optique de nous informer sur l'existence d'un biais dans notre modèle.
L'image ci-dessous montre les résultats du test.

\begin{figure}[H]
    \centering
    \includegraphics[width=.65\textwidth]{C:/Users/Emmanuel/Documents/latex-files/Photos/others/hétéroscédatiscity.jpg}
    \caption{Test de Breusch-Pagan}
\end{figure}
\noindent On obtient donc comme résultats : $\chi^2(1)=0.16$ et sa \textit{p-value} $\alpha'=0.6922$.
En comparant $\alpha'$ à 5\%, on conclut qu'il y a absence d'hétéroscédasticité car $\alpha'$ est 
largement supérieur à $\alpha = 5\%$.
\linebreak
\section{Autocorrélation des erreurs}
\begin{figure}[H]
    \centering
    \includegraphics[width=0.65\textwidth]{C:/Users/Emmanuel/Documents/latex-files/Photos/others/dwat~2.jpg}
    \caption{Test de Durbin-Watson}
\end{figure}

À l'issue du test de \texttt{"Durbin-Watson"} permis grâce à la méthode \colorbox{black}{\textcolor{white}{\texttt{estat dwatson}}}, 
nous avons obtenu une valeur qui n’est à la fois ni trop proche de 0 et de 2.
Par conséquent, le recours à la table de Durbin-Watson est nécessaire pour porter un jugement sur ce résultat.
Au risque de 10\%, on lit dans la table de \texttt{Durbin et Watson}\footnote{La table de \texttt{Durbin et Watson} a été téléchargée via ce lien : \url{https://gwenn.parent.free.fr/documents/dw.pdf}} à $k'=4$ et $n=29$ que $d_1 =1.12$ et
$d_2 = 1.74$. La valeur de la statistique DW (1.424164) se situe donc entre de $d_1$ et $d_2$, 
nous pouvons par conséquent supposer qu'il y a une absence d’autocorrélation des résidus.

\section{Estimation finale du modèle de régression linéaire multiple }
\begin{figure}[H]
    \centering
    \includegraphics[width=0.65\textwidth]{C:/Users/Emmanuel/Documents/latex-files/Photos/others/dernière regression.jpg}
    \caption{Nouvelle régression linéaire multiple}
\end{figure}
\begin{spacing}{1.3}

À cause de la non-normalité des erreurs, il était nécessaire de réestimer le modèle en tenant 
compte de cet aspect. Pour cela, nous avons utilisé la méthode \colorbox{black}{\textcolor{white}{\texttt{regress} [\textit{list var}], \texttt{robust}}}. On obtient le tableau ci-dessus
et le modèle reste par conséquent le même car les coefficients de la nouvelle régression sont identiques à la précédente.
\end{spacing}
\vspace*{-1cm}
\begin{multline*}
	Conso\_pub_t= -1.136147\ +\ 0.2566749\ Force\_travail_t\ ‐ \ 0.5691375\ IDE_t\ \\‐ \ 0.1695814\ inflation_t\ +\ \hat{\varepsilon}_t
\end{multline*}
\linebreak
Toutefois, l'on peut noter un changement au niveau des écarts types des résidus (\textit{colonne} \colorbox{black}{\textcolor{white}{\texttt{Std.Err}}}) et
 des autres valeurs du tableau.
%\newpage
\section{Interprétations}
%\enlargethispage{1\baselineskip}

\hspace{.8cm}Finalement, le modèle obtenu grâce à la régression multiple des données du tableau 1 est le suivant: 
\vspace*{-0.5cm}
\begin{multline*}
    Conso\_pub_t= -1.136147\ +\ 0.2566749\ Force\_travail_t\ ‐ \ 0.5691375\ IDE_t\ \\ ‐ \ 0.1695814\ inflation_t\ +\ \hat{\varepsilon}_t
\end{multline*}

Ce modèle reste la meilleure régression linéaire multiple décrivant les dépenses de consommation publique,
jusqu'à preuve du contraire. En outre, à l'issue de toutes les vérifications faites plus haut sur le modèle,
nous pouvons maintenant faire les interprétations suivantes :
\begin{itemize}[label=$\bullet$]
    \item Les dépenses de consommation publique s'expliquent à la fois de manière individuelle et globale par le taux d'emploi, les investissements directs étrangers et l'indice des prix à la consommation.
    \item La variation de $+25.67\%$ du taux d'emploi entraîne une variation de 1\% des dépenses de consommation publique.
    \item Cependant, la variation des investissements directs étrangers de $-56.91\%$ ou celle de l'indice des prix à la consommation de $‐16.96\%$ entraîne une variation de 1\% des dépenses de consommation publique.
    \item De manière générale, la variation des dépenses de consommation publique est causée à $48.72\%$ par le taux de l'emploi, les investissements directs étrangers et l'indice des prix à la consommation.
\end{itemize}

\end{document}







