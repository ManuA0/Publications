\newpage \section*{Exercice 2}
	
	\noindent 1) Calculer les profils-lignes, présenter les résultats dans un tableau, et les interpréter.
	
	Pour cela, il faudrait trouver les valeurs de $x$ et $y$ du tableau.
	
	Sachant que :
	\[
	\left\{
	\begin{aligned}
		& 16+12 + y+17+x=50 \\
		& 23+16+x+6+y=50
	\end{aligned}
	\right.
	\]
	
	On obtiendra alors,
	\[
	\left\{
	\begin{aligned}
		& 45+x+y=50 \\
		& 45+x+y=5
	\end{aligned}
	\right.
	\]
	
	Ainsi, on conclura que $x+y=5 \Rightarrow x=5-y$ Par conséquent, sachant encore que $4xy-y^2=0$ nous aurons :
	\begin{align}
		& 4(5-y)y-y^2=0  \nonumber \\
		& \Rightarrow 20y-5y^2-y^2 = 0 \nonumber \\
		& \Rightarrow 20y-5y^2=0 \nonumber \\
		& \Rightarrow y=0 \ \text{ou} \ y=\frac{20}{5}=4 \nonumber
	\end{align}
	
	Ainsi, nous obtiendrons $4x \times 4 -4^2=0 \Rightarrow x=1$
	
	Dès lors $x=1$ et $y=4$
	
	% Table generated by Excel2LaTeX from sheet 'EX2'
	\begin{table}[htbp]
		\centering
		\begin{tabular}{c|c|c|c|c}
			\cline{2-5}
			& 1      & 2      & 3      & Somme \\  \hline
			A      & 8      & 3      & 12     & 23 \\ \hline
			B      & 7      & 9      & 1      & 17 \\ \hline
			C      & 1      & 4      & 5      & 10 \\ \hline
			Somme  & 16     & 16     & 18     & 50 \\ \hline
		\end{tabular}%
		\caption{Niveau de diplôme le plus élevé du répondant}
	\end{table}%
	
% Table generated by Excel2LaTeX from sheet 'EX2'
\begin{table}[htbp]
	\centering
	\begin{tabular}{c|c|c|c|c}
		\cline{2-5}
		& 1      & 2      & 3      & Somme \\ \hline
		A      & 0,35   & 0,13   & 0,52   & 1,00 \\ \hline
		B      & 0,41   & 0,53   & 0,06   & 1,00 \\ \hline
		C      & 0,10   & 0,40   & 0,50   & 1,00 \\ \hline
		Somme  & 0,86   & 1,06   & 1,08   & 3,00 \\ \hline
	\end{tabular}%
	\caption{Tableau des profils-ligne}
\end{table}%

\noindent \underline{Interprétation} : 
\begin{itemize}
	\item  Parmi les personnes dont la tranche est de 15 à 24 ans, 35\% ont un diplôme inférieur au BAC, 13\% ont un BAC-BAC +3 et 52\% ont un master et plus.
	\item Parmi celles dont l'âge est compris entre 25 et 44 ans, 41\% ont un diplôme inférieur au BAC, 53\% ont un diplôme équivalent au BAC-BAC +3 et seulement 6\% ont un master et plus.
	\item  Enfin, parmi les personnes dont l'âge varie entre 45 ans et plus, 10\% ont un diplôme inférieur au BAC, 40\% ont un BAC-BAC +3 et 50\% ont un master et plus.
\end{itemize}
	\noindent 2) Calculer les profils-colonnes, présenter les résultats dans un tableau, et les interpréter.
	
% Table generated by Excel2LaTeX from sheet 'EX2'
\begin{table}[htbp]
	\centering
	\begin{tabular}{c|c|c|c}
		\cline{2-4}
		& 1      & 2      & 3 \\ \hline
		A      & 0,50   & 0,19   & 0,67 \\ \hline
		B      & 0,44   & 0,56   & 0,06 \\ \hline
		C      & 0,06   & 0,25   & 0,28 \\ \hline
		Somme  & 1,00   & 1,00   & 1,00 \\ \hline
	\end{tabular}%
	\caption{Tableau des profils-colonne}
\end{table}%

\noindent \underline{Interpretation} :
\begin{itemize}
	\item Parmi les individus qui ont un diplôme inférieur au BAC, 50\% ont un âge variant entre 15 et 24 ans, 44\% ont âge compris entre 25 et 44 ans et 6\% ont 45 ans et plus.
	\item Parmi les personnes diplômées d'un BAC-BAC +3, 19\% ont un âge compris entre 15 et 24, 56\% ont un âge compris entre 25 et 44 ans tandis que 25\% ont 45 ans et plus.
	\item Parmi les personnes ayant un master et plus, 67\% ont entre 15 et 24 ans, 6\% ont entre 25 et 44 ans et 28\% ont 45 ans et plus.
\end{itemize}
	
	\noindent 3) Calculer les contributions au $\chi^2$
	
	Pour effectuer cette tâche, nous calculerons les effectifs conjoints donnés par la formule suivante : $n_{ij}=\frac{\displaystyle n_{i \cdotp} \times n_{\cdotp j}}{\displaystyle n}$ Ce qui permet donc d'avoir le tableau ci-dessous :
	
	% Table generated by Excel2LaTeX from sheet 'EX2'
	\begin{table}[htbp]
		\centering
		\begin{tabular}{c|c|c|c|c}
			\cline{2-5}
			& 1      & 2      & 3      & Somme \\ \hline
			A      & 7,36   & 7,36   & 8,28   & 23,00 \\ \hline
			B      & 5,44   & 5,44   & 6,12   & 17,00 \\ \hline 
			C      & 3,20   & 3,20   & 3,60   & 10,00 \\ \hline
			Somme  & 16,00  & 16,00  & 18,00  & 50,00 \\ \hline
		\end{tabular}%
		\caption{Chi-Square Statistic Expected Values}
	\end{table}%
	
	Ensuite, nous faisons la différence entre les valeurs observées et les valeurs attendues en utilisant la formule suivante : $n_{ij}-\frac{\displaystyle n_{i\cdotp}\times n_{\cdotp j}}{\displaystyle n}$
	
% Table generated by Excel2LaTeX from sheet 'EX2'
\begin{table}[htbp]
	\centering
	
	\begin{tabular}{c|c|c|c|c}
		\cline{2-5}
		& 1      & 2      & 3      & Somme \\ \hline
		A      & 0,64   & -4,36  & 3,72   & 0,00 \\ \hline
		B      & 1,56   & 3,56   & -5,12  & 0,00 \\ \hline
		C      & -2,20  & 0,80   & 1,40   & 0,00 \\ \hline
		Somme  & 0,00   & 0,00   & 0,00   & 0,00 \\ \hline
	\end{tabular}%
	\caption{Observed Minus Expected Values}
\end{table}%

	
	Enfin, afin d'avoir la contribution au $\chi^2$, nous ferons donc : $\frac{\displaystyle \left(n_{ij}-\frac{\displaystyle n_{i\cdotp}\times n_{\cdotp j}}{n}\right)^2}{\displaystyle \frac{n_{i\cdotp}\times n_{\cdotp j}}{n}}$
	
% Table generated by Excel2LaTeX from sheet 'EX2'
\begin{table}[htbp]
	\centering
	\begin{tabular}{c|c|c|c|c}
		\cline{2-5}
		& 1      & 2      & 3      & Somme \\ \hline
		A      & 0,06   & 2,58   & 1,67   & 4,31 \\ \hline
		B      & 0,45   & 2,33   & 4,28   & 7,06 \\ \hline
		C      & 1,51   & 0,20   & 0,54   & 2,26 \\ \hline
		Somme  & 2,02   & 5,11   & 6,50   & 13,63 \\ \hline
	\end{tabular}%
	\caption{Contributions to the Total Chi-Square Statistic}
\end{table}%

	
	\noindent 4) Effectuer le test d’indépendance de V$_{1}$ et V$_{2}$. On donne $\chi_{5\%}^2 (4) = 9,48$; $\chi_{5\%}^2 (6) = 12,59$  ; $\chi_{5\%}^2 (8) = 15,51.$

La formulation du test est la suivante : 
\[
\begin{cases}
	\text{H}_0 : \text{$V_1$ et $V_2$ sont indépendantes}\\
	\text{H}_1 : \text{$V_1$ et $V_2$ sont dépendantes}
\end{cases}
\]
Sachant que $\displaystyle \chi_{5\%}({\text{calculé}})=13,63$ et que $\displaystyle \chi_{5\%} (L-1)(C-1)= \chi_{5\%} (4)=9.48$ alors on en déduit que les variables V$_1$ et V$_2$ sont dépendantes car $\displaystyle \chi_{5\%}({\text{calculé}}) > \chi_{5\%} (\text{théorique})$.

Par conséquent, les variables \guillemetleft tranches d'âge \guillemetright\ et \guillemetleft niveau de diplôme élevé du répondant \guillemetright\ sont liées.

	\noindent \textbf{}
	